

\serialNumberPage

%%%%% Abstracts and preface.
%
% Write the abstract(s) and the preface into a separate file for the sake of
% clarity. Pass the appropriate file name as the first argument to these
% commands. Put the \abstract in the primary language first and the
% \otherabstract in the secondary language second. Those who do not speak
% Finnish only need the first abstract. The second argument of the \preface
% command takes the place where the thesis was signed in.
%
% Edit the files tex/{use-of-ai,tekoalyn-kaytto}.tex to match your use of AI in
% generating this thesis.
%

\abstract

\otherabstract

\preface{frontmatter/alkusanat.tex}{Tampereella}

%%%%% Table of contents.


\hypersetup{
	linkcolor=black,
}

\tableofcontents

%%%%% Lists of figures, tables, listings and terms.
%
% Print the lists of figures and/or tables. Uncomment either of these commands
% as required. Both are optional, but if there are many important
% figures/tables, listing them may be a good idea.

\listoffigures
\listoftables
\lstlistoflistings

% Misc stuff related to how the glossary is displayed.

\glsaddall
\setglossarystyle{taulong}
\setlength{\glsnamewidth}{0.25\textwidth}
\setlength{\glsdescwidth}{0.75\textwidth}
\renewcommand*{\glsgroupskip}{}

% Print the default glossary of abbreviations, if necessary. Otherwise comment
% out. The appropriate Finnish variant is 'Lyhenteet'

\printglossary[title={Lyhenteet ja merkinnät}]

% Print more than one glossary with these lines. Otherwise comment out.

% \printglossary[type=symbs]
% \printglossary[type=label]
% ...

\listofpublications

% AI disclaimer page.

\aidisclaimerinclusioncmd
