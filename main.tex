%%%%%%%%%%%%%%%%%%%%%%%%%%%%%%%%%%%%%%%%%%%%%%%%%%%%%%%%%%%
%% Congratulations, you've made an excellent choice
%% of writing your Tampere University thesis using
%% the LaTeX system. This document attempts to be
%% as complete a template as possible to let you focus
%% on the most important part: the writing itself.
%% Thus the details regarding the visual appearance
%% and even structure have already been worked out
%% for you!
%%
%% I sincerely hope you will find this template useful
%% in completing your thesis project. I've tried to
%% add comments (followed by the % sign) to clarify
%% the structure and purpose of some of the commands.
%% Most of the magic happens in the file tauthesis.cls,
%% which you are more than welcome to take a look at.
%% Just refrain from editing it in the most crucial
%% versions of the thesis!
%%
%% I wish you and your thesis project the best of luck!
%% If this template causes you trouble along the way
%% or if you've any suggestions for improving it,
%% please be in contact through GitHub
%% (<URL HERE>)
%%
%% Yours,
%%
%% Ville Koljonen
%%
%% PS. This template or its associated class file don't
%% come with a warranty. The content is provided as is,
%% without even the implied promise of fitness to the
%% mentioned purpose. You, as the author of the thesis,
%% are responsible for the entire work, including the
%% provided material. No one else is liable to you for
%% any damage inflicted on you or your thesis, were it
%% caused by using this template or not.
%%%%%%%%%%%%%%%%%%%%%%%%%%%%%%%%%%%%%%%%%%%%%%%%%%%%%%%%%%%

%%%%% NOTICE %%%%%
%% Please read through the entire template
%% (files under ./tex) to find all instructions.
%% It is possible that the attached pdf files
%% do not include the latest information.
%%%%%%%%%%%%%%%%%%

%%%%% INSTRUCTIONS FOR COMPILING THE DOCUMENT %%%%%
%% Overleaf: just click Recompile.
%% Terminal:
%%  1. pdflatex main.tex
%%  2. makeindex -s main.ist -t main.glg -o main.gls main.glo
%%  3. biber main
%%  4. pdflatex main.tex
%%  5. pdflatex main.tex
%% Similar sequence of commands is also required
%% in LaTeX specific editors.
%%%%%%%%%%%%%%%%%%%%%%%%%%%%%%%%%%%%%%%%%%%%%%%%%%%

%%%%% METADATA %%%%%
%
% Always keep the following metadata up to date! This is important for your
% PDF file to comply to accessibility standards. (And yes, this information
% must remain here, before \documentclass[...]{...}.)

%
% metadata.tex
%
% Fill in your document metadata in this file.
% This is included into the file main.tex for you.
%
% Define your citation options here.
% Valid values for style and sorting options
% can be found in the BibLaTeX manual: https://ctan.org/pkg/biblatex.
%
% Valid values for thesis types are bachelor, master, licentiate and doctor for enthesistype,
% and kandidaatti, maisteri, diplomi, lisensiaatti and tohtori for fithesistype.
% The Finnish end English thesis types need to match.
%
% The rest of the inputs should be rather self-explanatory.

% This declares to LaTeX that it should use modern internale LaTeX
% features not available in LaTeX2e.

\DocumentMetadata{
  pdfversion=1.7,
  pdfstandard=A-2b,
  lang=fi,
  uncompress,
}

\documentclass[
  author={Firstname Lastname},
  citationsorting=nyt,
  citationstyle=ieee,
  enfacultyname={Faculty},
  enkeywords={keyword1, keyword2, ...},
  enprogrammename={Thesis programme},
  ensubtitle={Descriptive subtitle},
  enthesistype=doctor,
  entitle={Thesis template},
  fifacultyname={Tiedekunta},
  fikeywords={avainsana1, avainsana2, ...},
  fiprogrammename={Tutkinto-ohjelman nimi},
  fisubtitle={Kuvaava alaotsikko},
  fithesistype={tohtori},
  fititle={Opinnäytetyöpohja},
  includePublications=true,
  mainlanguage={finnish},
  subject={Metatietoihin tuleva työn lyhyt kuvaus},
]{tauthesis}


%%% preamble.tex
%
% This file is for including LaTeX libraries or packages and defining your own
% commands.
%
% NOTE: The glossaries package loaded by tauthesis.cls throws a warning: No
% language module detected for 'finnish'. You can safely ignore this. All
% other warnings should be taken care of, before your thesis is submitted!

%%%%% Your packages.
%
% Before adding packages, see if they can be found in tauthesis.cls already.
% If you're not sure that you need a certain package, don't include it in the
% document! This can dramatically reduce compilation time.

% Graphs
% \usepackage{pgfplots}
% \pgfplotsset{compat=1.15}

% Subfigures and wrapping text
% \usepackage{subcaption}

%% Theorem environments and their numbering.
%
% Define both English and Finnish theorem types. These all follow the same
% counter. See the documentation of amsthm to see how these can be changed to
% suit your needs, if necessary.
%

\usepackage{amsthm}

\theoremstyle{definition}

\newtheorem{definition}{Definition}[chapter]
\newtheorem{theorem}[definition]{Theorem}
\newtheorem{lemma}[definition]{Lemma}
\newtheorem{corollary}[definition]{Corollary}
\newtheorem{example}[definition]{Example}

\newtheorem{maaritelma}[definition]{Määritelmä}
\newtheorem{lause}[definition]{Lause}
\newtheorem{apulause}[definition]{Apulause}
\newtheorem{seurauslause}[definition]{Seurauslause}
\newtheorem{esimerkki}[definition]{Esimerkki}

% Mathematics packages
\usepackage{mathtools, amssymb}
%\usepackage{bm}

% Chemistry packages
% \usepackage{chemfig}
% \usepackage[version=4]{mhchem}

% Text hyperlinking
% \usepackage{hyperref}
% \hypersetup{hidelinks}

% (SI) unit handling
\usepackage{siunitx}

\sisetup{
    detect-all,
    math-sf=\mathrm,
    exponent-product=\cdot,
    output-decimal-marker={,} % for theses in FINNISH!
}

%% For code listings.

\usepackage{listings}

% This global code listing configuration is required for automatically
% replacing special characters with corresponding LaTeX commands, in code files
% included with \lstinputlisting. Without it, letters like 'ä' in code file
% comments will result in a LaTeX errors.
%
% Source: https://tex.stackexchange.com/a/574950.

\lstset{
    inputencoding = utf8,  % Input encoding
    extendedchars = true,  % Extended ASCII
    literate      =        % Support additional characters
        {á}{{\'a}}1  {é}{{\'e}}1  {í}{{\'i}}1 {ó}{{\'o}}1  {ú}{{\'u}}1
        {Á}{{\'A}}1  {É}{{\'E}}1  {Í}{{\'I}}1 {Ó}{{\'O}}1  {Ú}{{\'U}}1
        {à}{{\`a}}1  {è}{{\`e}}1  {ì}{{\`i}}1 {ò}{{\`o}}1  {ù}{{\`u}}1
        {À}{{\`A}}1  {È}{{\`E}}1  {Ì}{{\`I}}1 {Ò}{{\`O}}1  {Ù}{{\`U}}1
        {ä}{{\"a}}1  {ë}{{\"e}}1  {ï}{{\"i}}1 {ö}{{\"o}}1  {ü}{{\"u}}1
        {Ä}{{\"A}}1  {Ë}{{\"E}}1  {Ï}{{\"I}}1 {Ö}{{\"O}}1  {Ü}{{\"U}}1
        {â}{{\^a}}1  {ê}{{\^e}}1  {î}{{\^i}}1 {ô}{{\^o}}1  {û}{{\^u}}1
        {Â}{{\^A}}1  {Ê}{{\^E}}1  {Î}{{\^I}}1 {Ô}{{\^O}}1  {Û}{{\^U}}1
        {œ}{{\oe}}1  {Œ}{{\OE}}1  {æ}{{\ae}}1 {Æ}{{\AE}}1  {ß}{{\ss}}1
        {ẞ}{{\SS}}1  {ç}{{\c{c}}}1 {Ç}{{\c{C}}}1 {ø}{{\o}}1  {Ø}{{\O}}1
        {å}{{\aa}}1  {Å}{{\AA}}1  {ã}{{\~a}}1  {õ}{{\~o}}1 {Ã}{{\~A}}1
        {Õ}{{\~O}}1  {ñ}{{\~n}}1  {Ñ}{{\~N}}1  {¿}{{?`}}1  {¡}{{!`}}1
        {°}{{\textdegree}}1 {º}{{\textordmasculine}}1 {ª}{{\textordfeminine}}1
        {£}{{\pounds}}1  {©}{{\copyright}}1  {®}{{\textregistered}}1
        {«}{{\guillemotleft}}1  {»}{{\guillemotright}}1  {Ð}{{\DH}}1  {ð}{{\dh}}1
        {Ý}{{\'Y}}1    {ý}{{\'y}}1    {Þ}{{\TH}}1    {þ}{{\th}}1    {Ă}{{\u{A}}}1
        {ă}{{\u{a}}}1  {Ą}{{\k{A}}}1  {ą}{{\k{a}}}1  {Ć}{{\'C}}1    {ć}{{\'c}}1
        {Č}{{\v{C}}}1  {č}{{\v{c}}}1  {Ď}{{\v{D}}}1  {ď}{{\v{d}}}1  {Đ}{{\DJ}}1
        {đ}{{\dj}}1    {Ė}{{\.{E}}}1  {ė}{{\.{e}}}1  {Ę}{{\k{E}}}1  {ę}{{\k{e}}}1
        {Ě}{{\v{E}}}1  {ě}{{\v{e}}}1  {Ğ}{{\u{G}}}1  {ğ}{{\u{g}}}1  {Ĩ}{{\~I}}1
        {ĩ}{{\~\i}}1   {Į}{{\k{I}}}1  {į}{{\k{i}}}1  {İ}{{\.{I}}}1  {ı}{{\i}}1
        {Ĺ}{{\'L}}1    {ĺ}{{\'l}}1    {Ľ}{{\v{L}}}1  {ľ}{{\v{l}}}1  {Ł}{{\L{}}}1
        {ł}{{\l{}}}1   {Ń}{{\'N}}1    {ń}{{\'n}}1    {Ň}{{\v{N}}}1  {ň}{{\v{n}}}1
        {Ő}{{\H{O}}}1  {ő}{{\H{o}}}1  {Ŕ}{{\'{R}}}1  {ŕ}{{\'{r}}}1  {Ř}{{\v{R}}}1
        {ř}{{\v{r}}}1  {Ś}{{\'S}}1    {ś}{{\'s}}1    {Ş}{{\c{S}}}1  {ş}{{\c{s}}}1
        {Š}{{\v{S}}}1  {š}{{\v{s}}}1  {Ť}{{\v{T}}}1  {ť}{{\v{t}}}1  {Ũ}{{\~U}}1
        {ũ}{{\~u}}1    {Ū}{{\={U}}}1  {ū}{{\={u}}}1  {Ů}{{\r{U}}}1  {ů}{{\r{u}}}1
        {Ű}{{\H{U}}}1  {ű}{{\H{u}}}1  {Ų}{{\k{U}}}1  {ų}{{\k{u}}}1  {Ź}{{\'Z}}1
        {ź}{{\'z}}1    {Ż}{{\.Z}}1    {ż}{{\.z}}1    {Ž}{{\v{Z}}}1
}

%%%%% Your commands.

% Print verbatim LaTeX commands
\newcommand{\verbcommand}[1]{\texttt{\textbackslash #1}}

% Command for formatting code.

\newcommand\code[1]{\texttt{#1}}

% A delimiter command for the norm of a vector with mathtools.

\DeclarePairedDelimiter\norm{\lVert}{\rVert}

% Basic theorems in Finnish and in English.
% Remove [chapter] if you wish a simply
% running enumeration.
% \newtheorem{lause}{Lause}[chapter]
% \newtheorem{theorem}[lause]{Theorem}

% \newtheorem{apulause}[lause]{Apulause}
% \newtheorem{lemma}[lause]{Lemma}

% Use these versions for individually
% enumerated lemmas
% \newtheorem{apulause}{Apulause}[chapter]
% \newtheorem{lemma}{Lemma}[chapter]

% Definition style
% \theoremstyle{definition}
% \newtheorem{maaritelma}{Määritelmä}[chapter]
% \newtheorem{definition}[maaritelma]{Definition}
% examples in this style

%%%%% Glossary information.

% Use the following lines ONLY if you need more
% than one glossary. The first argument specifies
% a type label for the glossary and the second
% the displayed name.
% \newglossary*{symbs}{Symbols}
% \newglossary{label}{Displayed name}
% ...

\makeglossaries

% Use this line if using the default glossary.
% Otherwise comment out.

\loadglsentries[main]{tex/sanasto.tex}

% Use this line if using more than one glossary.
% Otherwise comment out.
% \loadglsentries[symbs]{tex/sanasto2.tex}

%%%%% Citation information.

% Commonly used bibliography modifications.
% Feel free to play around with them.

%\ExecuteBibliographyOptions{%
%sorting=none,
%maxbibnames=99,
%maxcitenames=2,
%giveninits=true,
%uniquename=init,
%sortcites,
%sortlocale=fin}

%\DeclareNameAlias{sortname}{last-first}
%\DeclareNameAlias{author}{last-first}

%\DeclareFieldFormat[%
%    article,inbook,incollection,inproceedings,
%    patent,thesis,unpublished]{citetitle}{#1\isdot}
%\DeclareFieldFormat[%
%    article,inbook,incollection,inproceedings,
%    patent,thesis,unpublished]{title}{#1\isdot}
%\DeclareFieldFormat{pagetotal}{#1 \bibstring{page}}

%\AtBeginBibliography{\renewcommand*{\makelabel}[1]{#1\hss}}

%\DefineBibliographyExtras{english}{\let\finalandcomma=\empty}

\addbibresource{tex/references.bib}
 % You can add packages and define new commands in this file.

\begin{document}

%%%%% FRONT MATTER %%%%%

%%%%% Thesis information and title page.

\titlepagematter

\maketitle

\frontmatter



\serialNumberPage

%%%%% Abstracts and preface.
%
% Write the abstract(s) and the preface into a separate file for the sake of
% clarity. Pass the appropriate file name as the first argument to these
% commands. Put the \abstract in the primary language first and the
% \otherabstract in the secondary language second. Those who do not speak
% Finnish only need the first abstract. The second argument of the \preface
% command takes the place where the thesis was signed in.
%
% Edit the files tex/{use-of-ai,tekoalyn-kaytto}.tex to match your use of AI in
% generating this thesis.
%

\abstract

\otherabstract

\preface{frontmatter/alkusanat.tex}{Tampereella}

%%%%% Table of contents.


\hypersetup{
	linkcolor=black,
}

\tableofcontents

%%%%% Lists of figures, tables, listings and terms.
%
% Print the lists of figures and/or tables. Uncomment either of these commands
% as required. Both are optional, but if there are many important
% figures/tables, listing them may be a good idea.

\listOfFigures

\listOfTables

\listOfListings

% Misc stuff related to how the glossary is displayed.

\glsaddall
\setglossarystyle{taulong}
\setlength{\glsnamewidth}{0.25\textwidth}
\setlength{\glsdescwidth}{0.75\textwidth}
\renewcommand*{\glsgroupskip}{}

% Print the default glossary of abbreviations, if necessary. Otherwise comment
% out. The appropriate Finnish variant is 'Lyhenteet'

\printglossary[title={Lyhenteet ja merkinnät}]

% Print more than one glossary with these lines. Otherwise comment out.

% \printglossary[type=symbs]
% \printglossary[type=label]
% ...

\listofpublications

% AI disclaimer page.

\aidisclaimerinclusioncmd


%%%%% MAIN MATTER %%%%%

\mainmatter

\hypersetup{
	linkcolor=taupurple,
}

% Actually include your chapters in the file specified by this input statement.

%
% index.tex
%
% This file is where you should include your actual content. Write each of
% the chapters of the thesis into a separate file for the sake of clarity.
% They can be \input as shown below. Give both the chapters and their files as
% descriptive names as possible.
%

\chapter{Introduction}
\label{ch:introduction}

This chapter introduces the research topic, providing background information and outlining the motivation behind the study. It also presents the research objectives and questions that guide the investigation.

\section{Background and Motivation}
\label{sec:background-and-motivation}

In recent years, the integration of Artificial Intelligence (AI) assistants into various software systems has gained significant attention. AI assistants have the potential to enhance user experience, streamline workflows, and improve decision-making processes. Product Lifecycle Management (PLM) systems, which manage the entire lifecycle of a product from inception to disposal\parencite{stark_product_2015}, can particularly benefit from the integration of AI assistants. By leveraging AI technologies, PLM systems can provide users with intelligent support, automate routine tasks, and allow for better collaboration among various stakeholders. For instance, AI applications in PLM span product design, manufacturing, and service stages, offering opportunities for innovation while presenting challenges such as data governance and algorithm reliability\parencite{wang_artificial_2021}.

In the case of PLM systems, one of the key challenges is the fact that they often handle vast amounts of documentation, including standard operating procedures (SOPs), manuals, and technical specifications. Managing this documentation effectively is crucial for ensuring that users can access the information they need when they need it. However, the explosive growth of unstructured and heterogeneous data in PLM systems can lead to fragmentation and integration difficulties\parencite{wang_artificial_2021}. This makes it challenging to maintain up-to-date and accurate documentation, which is essential for effective decision-making and operational efficiency. Because of the adaptability of PLM systems to various industries and organizational needs, high-quality documentation is critical for user training, compliance, and knowledge transfer since there is no one-size-fits-all solution for PLM implementations\parencite{stark_product_2015-1}. While PLM systems can be deployed in various environments\parencite{stark_product_2015-1}, they often operate in on-premises environments with strict data governance policies, which can further complicate the integration of AI assistants.

To manage and utilize this documentation effectively, there is a growing interest in developing AI assistants that can interact with PLM systems. These AI assistants can help users navigate the vast amounts of documentation, provide context-aware help for specific tasks, and allow for knowledge sharing among team members. To implement an effective context-aware AI assistant within a PLM system, it is essential to address several challenges, including data integration, natural language processing, and user interface design. One promising approach to building such AI assistants is the use of Retrieval-Augmented Generation (RAG) systems, which combine large language models (LLMs) with external knowledge bases to provide accurate and contextually relevant responses\parencite{lewis_retrieval-augmented_2021}.

%@TODO: expand this background section later with content below

\textit{I will add more information about Sovelia Core and its documentation management challenges here...}

\textit{Additionally, I will elaborate a bit on the research methodology and system design generally here...}

\section{Research Objectives and Questions}
\label{sec:research-objectives-and-questions}

The primary objective of this research is to develop an AI assistant that is seamlessly integrated with a PLM system, specifically Sovelia Core, to enhance user experience and improve documentation management. The AI assistant will leverage RAG techniques to provide context-aware assistance based on the extensive documentation from various sources stored within the database of the PLM system. To achieve this objective, the research will focus on the following research questions:

\begin{itemize}
	\item \textbf{RQ1 (Design)}: What RAG system architecture best fits Sovelia Core's on-prem, PostgreSQL-based environment and data governance constraints?
	\item \textbf{RQ2 (Process)}: How should customer documentation pipelines (parse $ \rightarrow $ chunk $ \rightarrow $ embed $ \rightarrow $ retire) be engineered for reliability and maintainability?
	\item \textbf{RQ3 (Quality \& Risk)}: Which risks (hallucinations, stale context, access control) are most critical, and what mitigations work in practice?
\end{itemize}

By addressing these research questions, this study aims to contribute to the development of effective AI assistants for PLM systems, ultimately enhancing user experience and improving documentation management in complex industrial environments.

In chapter~\ref{ch:literature-review}, a comprehensive overview of the relevant literature and technologies related to AI assistants, PLM systems, and RAG techniques is provided. Chapter~\ref{ch:research-methodology} outlines the research methodology employed in this study, including the design and implementation of the AI assistant. In chapter~\ref{ch:system-design-and-implementation}, the architectural choices and technical implementations made during the development of the AI assistant are detailed. In chapter~\ref{ch:evaluation-and-results}, the results of the experiments and evaluations are presented. Finally, chapter~\ref{ch:conclusion} concludes the thesis with a summary of contributions, limitations, and suggestions for future research.

%@TODO: write a proper summary of the chapters above once they are finalized

\textit{The summary above will be updated later to reflect the actual chapter titles and content once they are finalized...}


\chapter{Literature review}
\label{ch:literature-review}

This chapter establishes the theoretical and empirical foundation and background for integrating a resource-augmented AI assistant into Sovelia Core PLM. The review begins by examining the fundamental characteristics of Product Lifecycle Management systems and the role of AI assistants in data-intensive workflows, establishing why a human-in-the-loop considerable approach for enterprise PLM environments (\autoref{sec:plm-systems-and-ai-assistants}). After this, the overall architecture and principles of Retrieval-Augmented Generation (RAG) systems is explained briefly, focusing on how the combination of parametric and non-parametric memory addresses key limitations of purely parametric models in knowledge-intensive tasks (\autoref{sec:resource-augmented-ai-systems}).

The core of this review examines five recent case studies of RAG-based systems deployed in technical documentation and manufacturing contexts (\autoref{sec:case-studies-on-similar-integrations}): Heredia Álvaro and Barreda's ceramic tile manufacturing quality control system, Shejuti et al.'s technical documentation chatbot, Knollmeyer et al.'s manufacturing documentation system with knowledge graph enhancement, Wan et al.'s hybrid KG-vector RAG for smart manufacturing, and Wang et al.'s comprehensive framework for AI in PLM. These studies reveal both common success factors, particularly in domain-specific adaptation, and persistent challenges. The contexts of these studies are comparable to those of typical PLM system users, who are predominantly in the manufacturing industry \parencite{stark_product_2015}. By synthesizing lessons from these deployments, this review identifies key design considerations and risk factors to inform the architecture and implementation strategy of the AI assistant for Sovelia Core, especially given the challenges faced by companies in the manufacturing industry.

\section{Overview of PLM systems and AI assistants}
\label{sec:plm-systems-and-ai-assistants}

Product Lifecycle Management (PLM) systems are integral to managing the entire lifecycle of a product from inception, through engineering design and manufacturing, to service and disposal. They provide a centralized repository for all product-related information, facilitating collaboration among different departments and stakeholders \parencite{stark_product_2015}. In PLM systems generally the lifecycle states of product include: imagine, define, realise, support/use and retire/dispose \parencite{stark_product_2015-1}. All these separate stages generate data (including documentation) that needs to be managed effectively. The purpose of PLM system is to centralize all this data and provide users with access to the right information at the right time.

AI assistants, in the context of data-intensive workflows, can be defined as \textit{semi-automatic interactive tools} that guide analysts through specific tasks by recommending suitable transformations or actions that respect constraints obtained through interaction with the analyst \parencite{petricek_ai_2023}. Analysts in this context mean the human utilizing the AI assistant. Unlike fully automatic systems that attempt to solve problems without human intervention, or purely manual tools that require complete human control, AI assistants implement an iterative interaction pattern where the system makes an initial recommendation, the user can provide feedback through structured constraints or selections, and the system refines its recommendations accordingly. This human-in-the-loop approach combines the scalability and automation of machine learning with critical human insight, making it particularly suitable for tasks where edge cases and domain-specific knowledge are crucial \parencite{petricek_ai_2023}. In this thesis, the aim is to develop an AI assistant that can be integrated into PLM system to enhance user experience by providing context-aware support, streamlining routine data wrangling tasks, and improving access to information while maintaining human oversight and control. The AI assistant is meant to provide users with answers based on the documentation stored within the PLM system's database. However, the AI assistant in the context of this thesis is not fully automatic, as the users still have control over the final decisions and actions taken based on the AI assistant's recommendations. The purpose of the AI assistant is to inform and help with decision-making, rather than replacing human judgment entirely.

\section{Resource-augmented AI systems}
\label{sec:resource-augmented-ai-systems}

Resource-augmented AI systems, also known as \emph{retrieval-augmented generation} (RAG) systems, represent a paradigm shift in how artificial intelligence systems access and utilize knowledge. Unlike purely parametric models that store all knowledge implicitly within their parameters, resource-augmented systems combine parametric memory (the weights of a neural network) with non-parametric memory (external knowledge sources such as databases, document collections, or knowledge graphs) \parencite{lewis_retrieval-augmented_2021}. This hybrid architecture addresses several fundamental limitations of parametric-only models: the inability to easily update or revise stored knowledge, difficulty in providing provenance for predictions, and the tendency to generate hallucinated or factually incorrect content \parencite{lewis_retrieval-augmented_2021}. In simplified terms, RAG systems can be viewed as AI systems that "look up" relevant information from an external source before generating a response, rather than relying solely on learned patterns from training data and thus enabling more accurate, up-to-date, and contextually relevant outputs. This will result in less hallucinations compared to standard large language models (LLMs) \parencite{lewis_retrieval-augmented_2021}.

\begin{figure}[htbp]
    \centering
    \small{\resizebox{\textwidth}{!}{% generated by Plantuml 1.2025.9       
\definecolor{plantucolor0000}{RGB}{0,0,0}
\definecolor{plantucolor0001}{RGB}{255,255,255}
\begin{tikzpicture}[yscale=-1
,pstyle0/.style={color=black,fill=white,line width=1.0pt}
,pstyle1/.style={color=black,line width=1.0pt}
,pstyle2/.style={color=black,fill=black,line width=1.0pt}
,pstyle3/.style={color=black,line width=1.0pt,dash pattern=on 7.0pt off 7.0pt}
]
\node at (283.395pt,15pt)[below right,color=black,inner sep=0]{\textbf{Retrieval-Augmented Generation (RAG) System Architecture}};
\draw[pstyle0] (13.5pt,42pt) -- (80.69pt,42pt) arc(270:360:3.75pt)  -- (90.19pt,58pt) -- (712.5pt,58pt) arc(270:360:2.5pt)  -- (715pt,556.5pt) arc(0:90:2.5pt)  -- (13.5pt,559pt) arc(90:180:2.5pt)  -- (11pt,44.5pt) arc(180:270:2.5pt) ;
\draw[pstyle1] (11pt,58pt) -- (90.19pt,58pt);
\node at (15pt,44pt)[below right,color=black,inner sep=0]{\textbf{RAG System}};
\draw[pstyle0] (399pt,280pt) arc (180:270:5pt) -- (404pt,275pt) -- (678pt,275pt) arc (270:360:5pt) -- (683pt,280pt) -- (683pt,361pt) arc (0:90:5pt) -- (678pt,366pt) -- (404pt,366pt) arc (90:180:5pt) -- (399pt,361pt) -- cycle;
\node at (517.335pt,277pt)[below right,color=black,inner sep=0]{\textbf{Retriever}};
\draw[pstyle0] (435pt,441pt) arc (180:270:5pt) -- (440pt,436pt) -- (589pt,436pt) arc (270:360:5pt) -- (594pt,441pt) -- (594pt,522pt) arc (0:90:5pt) -- (589pt,527pt) -- (440pt,527pt) arc (90:180:5pt) -- (435pt,522pt) -- cycle;
\node at (488.87pt,438pt)[below right,color=black,inner sep=0]{\textbf{Generator}};
\draw[pstyle0] (43pt,97pt) ..controls (43pt,87pt) and (167.5pt,87pt) .. (167.5pt,87pt) ..controls (167.5pt,87pt) and (292pt,87pt) .. (292pt,97pt) -- (292pt,356pt) ..controls (292pt,366pt) and (167.5pt,366pt) .. (167.5pt,366pt) ..controls (167.5pt,366pt) and (43pt,366pt) .. (43pt,356pt) -- (43pt,97pt);
\draw[pstyle1] (43pt,97pt) ..controls (43pt,107pt) and (167.5pt,107pt) .. (167.5pt,107pt) ..controls (167.5pt,107pt) and (292pt,107pt) .. (292pt,97pt);
\node at (126.29pt,109pt)[below right,color=black,inner sep=0]{\textbf{Knowledge Base}};
\node at (100.13pt,119pt)[below right,color=black,inner sep=0]{\textbf{(Non-parametric Memory)}};
\draw[pstyle0] (288.5pt,488pt) arc (180:270:20pt) -- (308.5pt,468pt) -- (349.82pt,468pt) arc (270:360:20pt) -- (369.82pt,488pt) -- (369.82pt,488pt) arc (0:90:20pt) -- (349.82pt,508pt) -- (308.5pt,508pt) arc (90:180:20pt) -- (288.5pt,488pt) -- cycle;
\node at (301.91pt,478pt)[below right,color=black,inner sep=0]{Vector Store};
\node at (298.5pt,488pt)[below right,color=black,inner sep=0]{(Embeddings)};
\draw[pstyle0] (573pt,317pt) arc (180:270:5pt) -- (578pt,312pt) -- (654.32pt,312pt) arc (270:360:5pt) -- (659.32pt,317pt) -- (659.32pt,337pt) arc (0:90:5pt) -- (654.32pt,342pt) -- (578pt,342pt) arc (90:180:5pt) -- (573pt,337pt) -- cycle;
\node at (583pt,322pt)[below right,color=black,inner sep=0]{Query Encoder};
\draw[pstyle0] (442.5pt,317pt) arc (180:270:5pt) -- (447.5pt,312pt) -- (532.29pt,312pt) arc (270:360:5pt) -- (537.29pt,317pt) -- (537.29pt,337pt) arc (0:90:5pt) -- (532.29pt,342pt) -- (447.5pt,342pt) arc (90:180:5pt) -- (442.5pt,337pt) -- cycle;
\node at (452.5pt,322pt)[below right,color=black,inner sep=0]{Similarity Search};
\draw[pstyle0] (478pt,478pt) arc (180:270:5pt) -- (483pt,473pt) -- (565.37pt,473pt) arc (270:360:5pt) -- (570.37pt,478pt) -- (570.37pt,498pt) arc (0:90:5pt) -- (565.37pt,503pt) -- (483pt,503pt) arc (90:180:5pt) -- (478pt,498pt) -- cycle;
\node at (488pt,483pt)[below right,color=black,inner sep=0]{Language Model};
\draw[pstyle0] (67.5pt,154pt) arc (180:270:5pt) -- (72.5pt,149pt) -- (135.69pt,149pt) arc (270:360:5pt) -- (140.69pt,154pt) -- (140.69pt,174pt) arc (0:90:5pt) -- (135.69pt,179pt) -- (72.5pt,179pt) arc (90:180:5pt) -- (67.5pt,174pt) -- cycle;
\node at (77.5pt,159pt)[below right,color=black,inner sep=0]{Document 1};
\draw[pstyle0] (175.5pt,154pt) arc (180:270:5pt) -- (180.5pt,149pt) -- (243.69pt,149pt) arc (270:360:5pt) -- (248.69pt,154pt) -- (248.69pt,174pt) arc (0:90:5pt) -- (243.69pt,179pt) -- (180.5pt,179pt) arc (90:180:5pt) -- (175.5pt,174pt) -- cycle;
\node at (185.5pt,159pt)[below right,color=black,inner sep=0]{Document 2};
\draw[pstyle0] (67.5pt,317pt) arc (180:270:5pt) -- (72.5pt,312pt) -- (135.69pt,312pt) arc (270:360:5pt) -- (140.69pt,317pt) -- (140.69pt,337pt) arc (0:90:5pt) -- (135.69pt,342pt) -- (72.5pt,342pt) arc (90:180:5pt) -- (67.5pt,337pt) -- cycle;
\node at (77.5pt,322pt)[below right,color=black,inner sep=0]{Document 3};
\draw[pstyle0] (176pt,317pt) arc (180:270:5pt) -- (181pt,312pt) -- (199.34pt,312pt) arc (270:360:5pt) -- (204.34pt,317pt) -- (204.34pt,337pt) arc (0:90:5pt) -- (199.34pt,342pt) -- (181pt,342pt) arc (90:180:5pt) -- (176pt,337pt) -- cycle;
\node at (186pt,322pt)[below right,color=black,inner sep=0]{...};
\draw[pstyle0] (548pt,636pt) ellipse (8pt and 8pt);
\draw[pstyle1] (548pt,644pt) -- (548pt,671pt)(535pt,652pt) -- (561pt,652pt)(548pt,671pt) -- (535pt,686pt)(548pt,671pt) -- (561pt,686pt);
\node at (538.1pt,688pt)[below right,color=black,inner sep=0]{User};
\draw[pstyle0] (731.5pt,129pt) -- (731.5pt,199pt) -- (870.85pt,199pt) -- (870.85pt,139pt) -- (860.85pt,129pt) -- (731.5pt,129pt);
\draw[pstyle0] (860.85pt,129pt) -- (860.85pt,139pt) -- (870.85pt,139pt) -- (860.85pt,129pt);
\node at (738.355pt,134pt)[below right,color=black,inner sep=0]{External knowledge source};
\node at (737.5pt,144pt)[below right,color=black,inner sep=0]{(e.g., PLM documentation,};
\node at (740.715pt,154pt)[below right,color=black,inner sep=0]{technical manuals, SOPs)};
\node at (795.01pt,164pt)[below right,color=black,inner sep=0]{~};
\node at (741.285pt,174pt)[below right,color=black,inner sep=0]{Easily updatable without};
\node at (760.39pt,184pt)[below right,color=black,inner sep=0]{model retraining};
\draw[pstyle0] (740pt,458pt) -- (740pt,518pt) -- (882.2pt,518pt) -- (882.2pt,468pt) -- (872.2pt,458pt) -- (740pt,458pt);
\draw[pstyle0] (872.2pt,458pt) -- (872.2pt,468pt) -- (882.2pt,468pt) -- (872.2pt,458pt);
\node at (762.74pt,463pt)[below right,color=black,inner sep=0]{Parametric Memory};
\node at (751.27pt,473pt)[below right,color=black,inner sep=0]{(Neural network weights)};
\node at (804.935pt,483pt)[below right,color=black,inner sep=0]{~};
\node at (747.805pt,493pt)[below right,color=black,inner sep=0]{Conditions output on both};
\node at (746pt,503pt)[below right,color=black,inner sep=0]{query and retrieved content};
\draw[pstyle1] (562.16pt,636.4pt) ..controls (586.27pt,589.98pt) and (628.73pt,489.14pt) .. (586pt,420pt) ..controls (570.33pt,394.64pt) and (554.72pt,398pt) .. (526pt,390pt) ..controls (504.52pt,384.02pt) and (443.66pt,395.6pt) .. (426pt,382pt) ..controls (421.1175pt,378.24pt) and (417.4856pt,373.1338pt) .. (414.785pt,367.5539pt) ..controls (414.6162pt,367.2052pt) and (414.4511pt,366.8546pt) .. (414.2895pt,366.5023pt) ..controls (414.2491pt,366.4143pt) and (416.6833pt,371.7921pt) .. (416.6434pt,371.7039pt);
\draw[pstyle2] (414.1689pt,366.2379pt) -- (414.2366pt,376.0865pt) -- (416.231pt,370.7929pt) -- (421.5246pt,372.7872pt) -- (414.1689pt,366.2379pt) -- cycle;
\node at (605pt,483pt)[below right,color=black,inner sep=0]{User Query};
\draw[pstyle1] (410.4238pt,366.0457pt) ..controls (410.3895pt,366.2243pt) and (410.3542pt,366.4026pt) .. (410.3178pt,366.5806pt) ..controls (409.1531pt,372.2775pt) and (406.905pt,377.675pt) .. (403pt,382pt) ..controls (389.39pt,397.08pt) and (373.16pt,376.48pt) .. (358pt,390pt) ..controls (335.83pt,409.77pt) and (330.6147pt,439.7188pt) .. (329.4147pt,461.8588pt);
\draw[pstyle2] (329.09pt,467.85pt) -- (333.5712pt,459.0797pt) -- (329.3606pt,462.8573pt) -- (325.583pt,458.6467pt) -- (329.09pt,467.85pt) -- cycle;
\node at (359pt,391pt)[below right,color=black,inner sep=0]{Search similar};
\node at (364.165pt,401pt)[below right,color=black,inner sep=0]{embeddings};
\draw[pstyle1] (326.29pt,467.63pt) ..controls (321.475pt,433.17pt) and (311.0525pt,358.58pt) .. (301.76pt,292.0788pt) ..controls (299.4369pt,275.4534pt) and (297.1844pt,259.3337pt) .. (295.1078pt,244.4729pt) ..controls (294.0695pt,237.0425pt) and (293.9055pt,235.8691pt) .. (292.9683pt,229.1623pt);
\draw[pstyle2] (292.138pt,223.2201pt) -- (289.422pt,232.687pt) -- (292.8299pt,228.172pt) -- (297.345pt,231.5799pt) -- (292.138pt,223.2201pt) -- cycle;
\node at (314pt,317pt)[below right,color=black,inner sep=0]{Retrieve top-k};
\node at (321.97pt,327pt)[below right,color=black,inner sep=0]{documents};
\draw[pstyle1] (292.0848pt,168.4581pt) ..controls (292.5082pt,168.6482pt) and (292.9461pt,168.846pt) .. (293.3979pt,169.0515pt) ..controls (295.2052pt,169.8734pt) and (297.2355pt,170.8183pt) .. (299.455pt,171.8842pt) ..controls (308.3331pt,176.1478pt) and (320.24pt,182.3469pt) .. (333.0225pt,190.3513pt) ..controls (358.5875pt,206.36pt) and (387.655pt,229.59pt) .. (403pt,259pt) ..controls (404.6713pt,262.2025pt) and (405.9964pt,265.6119pt) .. (407.0348pt,269.1311pt) ..controls (407.554pt,270.8906pt) and (408.0015pt,272.6777pt) .. (408.3847pt,274.48pt) ..controls (408.4086pt,274.5927pt) and (407.2107pt,268.8311pt) .. (407.2342pt,268.9438pt);
\draw[pstyle2] (408.4558pt,274.8182pt) -- (410.5396pt,265.1923pt) -- (407.4378pt,269.9229pt) -- (402.7071pt,266.8211pt) -- (408.4558pt,274.8182pt) -- cycle;
\node at (400.485pt,230pt)[below right,color=black,inner sep=0]{Retrieved};
\node at (397pt,240pt)[below right,color=black,inner sep=0]{Documents};
\draw[pstyle1] (420.439pt,366.2539pt) ..controls (420.4993pt,366.426pt) and (420.5597pt,366.5983pt) .. (420.6201pt,366.7708pt) ..controls (420.7409pt,367.1156pt) and (420.8619pt,367.4611pt) .. (420.9831pt,367.807pt) ..controls (421.2255pt,368.4988pt) and (421.4684pt,369.1925pt) .. (421.7118pt,369.8875pt) ..controls (422.1985pt,371.2775pt) and (422.6869pt,372.6725pt) .. (423.175pt,374.0675pt) ..controls (430.985pt,396.3875pt) and (438.74pt,418.695pt) .. (439pt,420pt) ..controls (439.7075pt,423.5425pt) and (440.3023pt,427.2095pt) .. (440.8014pt,430.919pt) ..controls (440.9262pt,431.8463pt) and (441.045pt,432.7764pt) .. (441.1581pt,433.7077pt) ..controls (441.2146pt,434.1734pt) and (441.2697pt,434.6395pt) .. (441.3234pt,435.1057pt) ..controls (441.3503pt,435.3388pt) and (440.7079pt,429.6094pt) .. (440.734pt,429.8425pt);
\draw[pstyle2] (441.403pt,435.8051pt) -- (444.3746pt,426.4153pt) -- (440.8455pt,430.8363pt) -- (436.4245pt,427.3072pt) -- (441.403pt,435.8051pt) -- cycle;
\node at (458.035pt,391pt)[below right,color=black,inner sep=0]{Query +};
\node at (437pt,401.1pt)[below right,color=black,inner sep=0]{Retrieved Context};
\draw[pstyle1] (448.3779pt,527.354pt) ..controls (448.4338pt,527.6107pt) and (448.4902pt,527.8679pt) .. (448.5472pt,528.1255pt) ..controls (448.6612pt,528.6409pt) and (448.7775pt,529.1581pt) .. (448.896pt,529.6771pt) ..controls (449.8445pt,533.8285pt) and (450.9404pt,538.0877pt) .. (452.2055pt,542.3453pt) ..controls (454.7356pt,550.8606pt) and (457.9425pt,559.37pt) .. (462pt,567pt) ..controls (480.43pt,601.67pt) and (509.9206pt,629.6815pt) .. (529.1806pt,646.0715pt);
\draw[pstyle2] (533.75pt,649.96pt) -- (529.4882pt,641.081pt) -- (529.9422pt,646.7196pt) -- (524.3035pt,647.1735pt) -- (533.75pt,649.96pt) -- cycle;
\node at (482pt,576pt)[below right,color=black,inner sep=0]{Generated};
\node at (484.29pt,586pt)[below right,color=black,inner sep=0]{Response};
\draw[pstyle3] (293.2287pt,164pt) ..controls (293.8106pt,164pt) and (294.3925pt,164pt) .. (294.9744pt,164pt) ..controls (296.1383pt,164pt) and (297.3021pt,164pt) .. (298.4659pt,164pt) ..controls (303.1212pt,164pt) and (307.7765pt,164pt) .. (312.4318pt,164pt) ..controls (321.7423pt,164pt) and (331.0528pt,164pt) .. (340.3633pt,164pt) ..controls (358.9843pt,164pt) and (377.6051pt,164pt) .. (396.2258pt,164pt) ..controls (433.4672pt,164pt) and (470.7081pt,164pt) .. (507.9488pt,164pt) ..controls (582.43pt,164pt) and (656.91pt,164pt) .. (731.39pt,164pt);
\draw[pstyle3] (451.1901pt,435.6097pt) ..controls (451.4022pt,435.0226pt) and (451.6198pt,434.437pt) .. (451.8429pt,433.8535pt) ..controls (453.628pt,429.1855pt) and (455.7711pt,424.6478pt) .. (458.3434pt,420.4494pt) ..controls (463.4881pt,412.0525pt) and (470.35pt,405.0125pt) .. (479.5pt,401pt) ..controls (523.46pt,381.72pt) and (649.98pt,385.77pt) .. (695.5pt,401pt) ..controls (729.07pt,412.23pt) and (760.7pt,437.48pt) .. (782.34pt,457.9pt);
\end{tikzpicture}
}}
    \caption{Retrieval-Augmented Generation (RAG) system architecture}
    \label{fig:rag-architecture}
\end{figure}

In a typical RAG architecture, the system consists of two main components: a \emph{retriever} that identifies relevant documents or passages from an external knowledge source given an input query, and a \emph{generator} that produces outputs conditioned on both the input and the retrieved content \parencite{lewis_retrieval-augmented_2021}. The retriever typically employs dense passage retrieval methods using bi-encoder architectures that compute semantic similarity between queries and documents in a shared embedding space, while the generator is commonly implemented as a pre-trained sequence-to-sequence transformer model. Crucially, these components can be trained end-to-end, allowing the retrieval mechanism to learn what information is relevant for the downstream task without requiring explicit retrieval supervision. \autoref{fig:rag-architecture} illustrates this architecture, showing how user queries flow through the retriever to access the knowledge base, and how retrieved context is combined with the query in the generator to produce responses.

\textcite{lewis_retrieval-augmented_2021} have highlighted that the advantages of resource-augmented systems are particularly pronounced in knowledge-intensive tasks. There are tasks that humans could not reasonably perform without access to external knowledge sources. By maintaining knowledge in an explicit, inspectable, and easily updatable non-parametric form, these systems enable dynamic knowledge updates by simply replacing or modifying the external knowledge source (a process sometimes called "index hot-swapping") without requiring costly model retraining and allowing for customizability for different needs. Additionally, the retrieved documents allow users to understand what information informed the system's response. In the context of PLM systems, this architecture is particularly well-suited for handling the extensive, evolving documentation ecosystem that characterizes enterprise software deployments. The customizable nature of RAG systems allow for tailoring the knowledge and thus the responses to the specific configurations and processes of different organizations using the PLM system.

\section{Case studies on similar integrations}
\label{sec:case-studies-on-similar-integrations}

Several recent implementations of RAG-based systems for technical documentation and manufacturing environments provide valuable insights for the design and development of an AI assistant for Sovelia Core, the main topic of this thesis. This section examines four representative case studies that demonstrate both the feasibility and challenges of integrating RAG technology into similarly complex technical domains.

\subsection*{Manufacturing quality control: Heredia Álvaro and Barreda's ceramic tile RAG system}

Heredia Álvaro and Barreda (2025) developed an advanced RAG system for manufacturing quality control in the ceramic tile industry \parencite{heredia_alvaro_advanced_2025}, addressing challenges comparable to those in PLM environments. Notably, as any companies in the manufacturing industry, ceramic tile manufacturers can be seen as potential PLM system users and thus, face similar issues with knowledge silos in technical documentation. In the study there are many similar characteristics mentioned: extensive technical documentation (defect handbooks, process articles), multiple specialized stages requiring expert knowledge (pressing, drying, enameling, firing, finishing), and the need for rapid procedural access for defect diagnosis and root cause analysis by users with varying expertise levels.

The system architecture detailed in the study employed a straightforward RAG implementation using OpenAI's text-embedding-3-large model for document indexing, with a pre-processing pipeline that structured 221 samples of ceramic defect information (defect type, identification methods, causes, solutions, origin areas). Documents were processed through validation mechanisms, with short documents fed directly to the LLM and longer documents first fragmented and stored in a Chroma vector store. Notably, Heredia Álvaro and Barreda implemented a two-stage retrieval approach: initial retrieval using a bi-encoder with Euclidean distance similarity, followed by post-retrieval reranking using a cross-encoder (sentence transformers ms-frame-MiniLM-L-6-v2) to prioritize the most relevant information samples. This combination addresses the computational trade-offs between efficiency (bi-encoder) and accuracy (cross-encoder). The generator itself used OpenAI's gpt-3.5-turbo-instruct model with customized prompts to make sure that responses prioritized external knowledge over the model's parametric memory. \parencite{heredia_alvaro_advanced_2025}

The system demonstrated practical effectiveness after the evaluation using both retrieval and generation metrics. The retrieval phase achieved 92.68\% Jaccard similarity and 85.81\% F1-score with optimal hyperparameters (k=7 retrieved samples, mean relevance score threshold), indicating high precision in identifying relevant context. The generation phase evaluation using ROUGE-L metrics yielded a mean score of 0.6108 (standard deviation 0.1371), significantly outperforming random baselines (mean 0.2300). Qualitative comparison against GPT-4 without domain adaptation demonstrated the RAG system's superiority in providing accurate, domain-specific answers. For example, when queried about carbon particles in ceramic tiles, the RAG system correctly identified the defect as impurities in raw materials and recommended weathering clays or finer sieving, whereas GPT-4 incorrectly attributed the issue to general fouling and cleaning problems. The system operates at low computational cost (approximately \$0.0012 per query) and achieved practical deployment for use cases including customer claim resolution, non-conformities reporting, and continuous improvement actions. \parencite{heredia_alvaro_advanced_2025}

For the Sovelia Core PLM integration, Heredia Álvaro and Barreda's work provides directly applicable insights aligned with this thesis's scope. Their implementation represents a pragmatic approach prioritizing foundational RAG capabilities over advanced techniques. The architecture demonstrates several design principles relevant to enterprise PLM deployment: (1) careful pre-processing and post-processing to optimize retrieval quality without requiring model fine-tuning, (2) systematic hyperparameter optimization through evaluation metrics, (3) structured knowledge representation (defect types, causes, solutions) that is similar to PLM documentation patterns, and (4) explicit validation mechanisms to ensure query relevance. The study's emphasis on using pre-trained embedding models without extensive fine-tuning, combined with the two-stage retrieval strategy, could be used when making architectural decisions for Sovelia Core. The authors' explicit evaluation methodology of measuring both retrieval precision and generation quality provides a replicable framework for validating the PLM assistant's performance. This case study confirms the feasibility of achieving practical value with relatively simple, maintainable RAG architectures before considering more complex extensions.

\subsection*{Technical documentation: Shejuti et al.'s MODTRAN chatbot}

Shejuti et al. (2025) addressed the problem of navigating extensive technical documentation through their RAG-based chatbot for MODTRAN (Moderate resolution atmospheric TRANsmission) software, a domain characterized by complex scientific documentation similar to specialized PLM system manuals \parencite{shejuti_extended_2025}. The document selection included a MODTRAN6 user manual, algorithm theoretical basis document (ATBD), and MODTRAN FAQ resources. This documentation complexity is somewhat comparable to enterprise PLM deployments with multiple document types serving different user needs.

The system architecture employed PyPDF2 for PDF text extraction while preserving hierarchical structure, BeautifulSoup for HTML parsing of FAQ pages, and CharacterTextSplitter to segment documents into 1000-character chunks with 200-character overlap. Embeddings were generated using HuggingFace's SciBERT-NLI model specifically selected for scientific document understanding, and stored in a FAISS vector database. The RAG pipeline retrieved top-k=13 chunks based on cosine similarity, passing them to an LLM through LangChain's load\_qa\_chain.

The system was evaluated by generating a set of query-and-answer pairs by a domain expert and then comparing the LLM responses against the expert's. Conclusion was that the LLM-generated answers were mostly accurate and relevant compared to the one's provided by the expert. Additionally, comparative testing against ChatGPT showed that the domain-adapted RAG system produced more concise and focused answers that aligned with expert expectations, whereas general-purpose LLMs provided more verbose but less specific responses.

For Sovelia Core PLM, this case study again highlights the effectiveness of relatively simple retrieval architectures with modern pre-trained embedding models, and the importance of chunk size and overlap parameters in balancing context completeness with retrieval precision. The study also emphasizes the need for iterative refinement based on expert evaluation, as initial implementations may suffer from retrieval noise that affects response quality. All in all, this case study is highly relevant to Sovelia Core PLM due to the similarity in documentation complexity and user needs.

\subsection*{Manufacturing documentation: Knollmeyer et al.'s Document GraphRAG}

Knollmeyer et al. (2025) introduced Document GraphRAG, a novel framework that enhances RAG systems by incorporating knowledge graphs built upon a document's intrinsic structure into the retrieval pipeline, specifically targeting manufacturing domain documentation \parencite{knollmeyer_document_2025}. This study is particularly relevant to PLM integration as it addresses persistent challenges in retrieval precision and context selection that hinder RAG effectiveness in technical documentation environments. Notably, the research employs Design Science Research methodology, which is the same methodological approach used in this thesis, to design, implement, and evaluate the GraphRAG framework.

The system architecture leverages graph-based document structuring with a keyword-based semantic linking mechanism to improve retrieval quality beyond naive RAG baselines. Unlike the aforementioned traditional RAG systems that treat documents as flat collections of chunks, Document GraphRAG constructs knowledge graphs that preserve hierarchical document structure, section relationships, and semantic connections between content elements. The framework maintains the simplicity of text-based retrieval while adding structural awareness through graph traversal, without the need for model fine-tuning. The implementation focuses on task-dependent optimizations for fundamental parameters: chunk size, keyword density, and top-k retrieval depth. \parencite{knollmeyer_document_2025}

Evaluation on established datasets (SQuAD, HotpotQA) and a new manufacturing dataset showed consistent improvements over naive RAG baselines in retrieval and generation metrics. GraphRAG particularly enhanced context relevance for queries requiring multi-section reasoning, making it suitable for manufacturing and PLM tasks that synthesize information from related sections (e.g., process specifications and technical requirements). The manufacturing dataset confirmed its effectiveness in domain-specific question answering, with better retrieval robustness. \parencite{knollmeyer_document_2025}

For Sovelia Core PLM integration, Knollmeyer et al.'s work provides connection between theoretical RAG concepts and practical manufacturing application. The study's focus on improving retrieval precision through structural awareness rather than complex model adaptations aligns well with this thesis's pragmatic approach. Key contributions relevant to Sovelia Core include: (1) the demonstration that document structure can enhance retrieval without requiring fine-tuning, (2) systematic evaluation on manufacturing-specific datasets that validate domain transferability, (3) evidence that reasoning capabilities matter for technical documentation (PLM users might need answers that span multiple documentation sections), and (4) task-dependent parameter optimization strategies applicable to PLM deployment. The framework's emphasis on knowledge graph construction from document structure suggests a natural evolution path for PLM systems, which already maintain structured information (product hierarchies, bill-of-materials relationships, configuration dependencies). While full graph-based retrieval represents an advanced enhancement, the study confirms that foundational RAG approaches remain effective for manufacturing documentation, with graph structures offering incremental improvements for complex queries rather than fundamental architectural requirements.

\subsection*{Smart manufacturing Q\&A: Wan et al.'s hybrid KG-Vector RAG system}

Wan et al. (2025) introduced a hybrid knowledge graph (KG)-vector RAG framework specifically designed for domain-centric question answering in smart manufacturing, addressing the precision-scalability trade-off inherent in conventional RAG approaches \parencite{wan_empowering_2025}. This research is particularly relevant to PLM integration as it discusses the domain gap and outdated knowledge challenges that LLMs face in specialized manufacturing contexts. The study recognizes that conventional vector-based RAG delivers rapid responses but might lead to contextually vague results, while knowledge graph methods can offer structured and relational reasoning at the expense of scalability and efficiency. This particularly interesting and applicable to enterprise PLM deployments where both precision and performance are of high-importance.

The system architecture implements a three-stage hybrid approach that systematically integrates structured and unstructured knowledge representations. First, a metadata-including knowledge graph was constructed from documentation through systematic extraction and indexing of structured information to capture domain-specific relationships (entities, concepts, and their connections). Second, semantic alignment was achieved by injecting domain-specific constraints to enhance the contextual relevance of knowledge representations, making sure that retrieved information is aligned with manufacturing terminology. Third, a layered hybrid retrieval strategy combined the knowledge graph's explicit reasoning with the vector-based method's search power. The results were integrated through prompt engineering to produce comprehensive and context-aware responses. \parencite{wan_empowering_2025}

Evaluated on design for additive manufacturing (DfAM) tasks, the hybrid approach achieved 77.8\% exact match accuracy and 76.5\% context precision, which demonstrates improvements over baseline of vector-only and KG-only approaches. The experimental results indicated that integrating structured knowledge graph information with vector-based retrieval and prompt engineering can enhance retrieval accuracy, contextual relevance, and efficiency in LLM-based question-answering systems in smart manufacturing area. \parencite{wan_empowering_2025}

For Sovelia Core PLM integration, Wan et al.'s work provides insights into more advanced hybrid architectures that further balance precision and scalability. However, the implementation complexity makes this approach a future enhancement candidaterather than an initial deployment priority. While these techniques demonstrated clear performance benefits, they introduce massive architectural complexity compared to more basic vector-based RAG implementation. For the scope of this thesis, the study's key contribution is that manufacturing-specific knowledge representation significantly improves question-answering precision. The research confirms that structured knowledge (which PLM systems inherently possess through product hierarchies, part relationships, and configuration data) can enhance retrieval when properly integrated. The precision-scalability trade-off identified by Wan et al. provides guidance for evaluating when increased architectural complexity justifies performance gains in enterprise PLM contexts.

\subsection*{AI in PLM: Wang et al.'s comprehensive review}

Wang et al. (2021) provided a comprehensive review of AI applications throughout the product lifecycle, examining how intelligent systems can support PLM in the design, manufacturing, and service stages \parencite{wang_artificial_2021}. While their review covers AI technologies broadly rather than RAG systems specifically, it establishes the wider context in which RAG-based documentation assistants operate within PLM environments.

The review identifies several AI applications relevant to understanding where RAG systems fit in the PLM landscape. In the design stage, AI supports market analysis through data mining, rapid conceptual design using case libraries, and design parameter recommendations through expert systems integrated with CAD platforms. The manufacturing stage uses AI for supplier selection, production planning and scheduling, and quality inspection using deep learning and machine vision. The service stage includes intelligent customer service, product status monitoring, and failure prediction using various machine learning approaches.

The service stage applications are particularly relevant for PLM documentation support. Wang et al. describe how intelligent customer service systems combine semantic retrieval using natural language processing to build knowledge bases with expert systems that match knowledge to user queries. This combination closely resembles modern RAG architectures, where retrieval systems identify relevant documentation and generation systems provide contextual answers.

Several challenges identified by Wang et al. directly affect RAG deployment in PLM environments. They emphasize that data quality, algorithm interpretability, and security remain critical concerns for AI in manufacturing contexts. The need to integrate data across different PLM stages and systems presents a significant challenge, as does the requirement to make AI systems understandable and trustworthy for industrial users who may be skeptical of AI recommendations in engineering workflows.

For Sovelia Core, this review highlights important considerations beyond basic document retrieval. A RAG assistant should eventually integrate with other PLM systems, support queries across different lifecycle stages, and provide transparent explanations to build user trust. The emphasis on security and data sovereignty in industrial contexts reinforces the importance of on-premise deployment for PLM environments. While Wang et al.'s study from 2021 does not cover recent RAG developments, the fundamental challenges they identify around data integration, transparency, and trust remain highly relevant for current PLM AI implementations.

\section{Synthesis and implications for Sovelia Core PLM}

Collectively, the case studies detailed in section \ref{sec:case-studies-on-similar-integrations} demonstrate both the technical feasibility and practical challenges of integrating RAG-based AI assistants into complex technical documentation environments. For the scope of this thesis, which focuses on developing a foundational text-based RAG system, the case studies reveal several directly applicable success factors:

\begin{enumerate}
    \item The effectiveness of modern pre-trained embedding models without requiring extensive fine-tuning
    \item The importance of fundamental retrieval parameters (chunk size, overlap, top-k selection) in balancing context chunk completeness with precision
    \item The value of structured document representation (knowledge graphs, hierarchical organization) for further improving retrieval precision
    \item Iterative refinement based on expert user feedback and evaluation metrics
\end{enumerate}

The case studies also highlight more advanced techniques that demonstrate significant performance improvements but introduce substantial implementation complexity. Advanced retrieval strategies such as graph-based traversal (\textcite{knollmeyer_document_2025}) and hybrid KG-vector architectures (\textcite{wang_artificial_2021}) show promise for more complex reasoning tasks and precision-critical applications, while multimodal integration and parameter-efficient fine-tuning techniques from other manufacturing AI research require specialized infrastructure and manual annotation overhead. While these approaches show potential benefits, their requirements for knowledge graph construction, domain-specific ontology definition, manual annotation, specialized training infrastructure, and domain-specific model adaptation position them as future enhancement opportunities rather than initial implementation priorities. The persistent challenges identified across implementations, including retrieval precision issues, difficulty handling complex multi-condition queries, the precision-scalability trade-off, and the need for domain-specific evaluation datasets, further reinforce the rationale for establishing a simpler, more maintainable text-based RAG foundation before considering advanced extensions.

For Sovelia Core PLM's initial deployment, these insights suggest prioritizing:

\begin{enumerate}
    \item A straightforward retrieval architecture using pre-trained embedding models
    \item Careful tuning of fundamental parameters (chunking strategy, retrieval depth, keyword density)
    \item Leveraging existing PLM document structure (hierarchies, relationships) where possible
    \item Evaluation using both retrieval and generation metrics
    \item Human-in-the-loop feedback mechanisms for continuous improvement
\end{enumerate}

The on-premise deployment constraint necessitates particular attention to resource-efficient architectures and local data governance. Advanced retrieval strategies such as graph-based traversal, hybrid KG-vector architectures, multimodal capabilities, and parameter-efficient fine-tuning remain valuable directions for future iterations once the baseline text-based system demonstrates practical value and achieves production stability.


\chapter{Research methodology}
\label{ch:research-methodology}

In this chapter, we detail the research methodology employed in this study, focusing on the design science research (DSR) paradigm and the single case study approach. We describe the case context of Sovelia Core, an on-premise PLM system, and outline the data collection and evaluation strategies used to assess the developed Retrieval-Augmented Generation (RAG) system.

\section{Design science methodology}
\label{sec:design-science-methodology}

Design Science Research (DSR) is a problem-solving paradigm that seeks to create and evaluate artifacts intended to solve identified organizational problems \parencite{johannesson_introduction_2021}. Unlike explanatory science, which aims to understand and explain phenomena, design science is prescriptive as it produces knowledge on how to design and build artifacts that serve human purposes. In the context of information systems research, these artifacts can take various forms, including constructs, models, methods, and instantiations.

According to \textcite{johannesson_introduction_2021}, design science research follows a structured process consisting of five key activities: (1) \emph{explicate problem}, where the practical problem and its context are identified and defined; (2) \emph{define requirements}, which establishes the criteria that the artifact must satisfy; (3) \emph{design and develop artifact}, involving the actual construction of the solution; (4) \emph{demonstrate artifact}, where the artifact's utility is shown in solving the problem; and (5) \emph{evaluate artifact}, assessing how well it meets the requirements and solves the problem.

In this thesis, we employ design science methodology to develop an AI assistant integrated with Sovelia Core, an on-premise PLM system. The practical problem we address is the knowledge accessibility challenge faced by users navigating fragmented documentation across vendor manuals, release notes, and customer-specific standard operating procedures. The artifact in this case is the RAG system, which is designed, implemented, and evaluated through iterative cycles within both internal and external contexts. Internal environments, such as testing sandboxes, allow for rapid testing and refinement, while external contexts are mainly gathered through user feedback from demos and pilot implementations. The purpose of this approach is to contribute both a practical solution for Sovelia Core and generalizable knowledge about RAG system integration in on-premise enterprise PLM environments.

\section{Case study approach}
\label{sec:case-study-approach}

This research employs a single case study design \parencite{yin_case_2018} to investigate the integration of a RAG-based AI assistant within an on-premise PLM environment. The case study approach allows for in-depth exploration of the technical, organizational, and practical challenges inherent in deploying enterprise AI systems under strict data governance constraints.

\subsection{Case description: Sovelia Core PLM environment}
\label{subsec:case-description}

Sovelia Core is an on-premise PLM platform specifically beneficial for medium to large manufacturing organizations. It is developed and maintained by Symetri, a provider of technology solutions catering to design, engineering, construction, and manufacturing businesses \parencite{noauthor_symetri_nodate}. The main purpose of the system is to serve as a centralized hub for managing product data, engineering change processes, bill-of-materials structures, and document management workflows throughout the product lifecycle. The system has a history of more than two decades and is widely adopted among various manufacturing enterprises in the Nordic region \parencite{noauthor_about_nodate}.

\subsubsection{Technical architecture and deployment model}

Sovelia Core follows a traditional client-server architecture deployed entirely within customer premises. The system is built on a custom technology stack, with a PostgreSQL database serving as the primary data repository. Client applications connect to the server through proprietary APIs, and the system can be integrated with both CAD systems (e.g. Autodesk Inventor) and various ERP platforms.

The on-premise deployment model is not just a technical preference but usually a fundamental requirement for Sovelia Core's customers. Manufacturing organizations in the mechanical engineering, industrial automation, and equipment manufacturing sectors typically handle sensitive intellectual property in the form of CAD models, proprietary designs, and confidential customer specifications. These organizations often operate under strict data sovereignty requirements, regulatory compliance mandates, or even contractual obligations that prohibit storing product data on external cloud infrastructure. Consequently, any AI-powered solution integrated with Sovelia Core must respect these constraints and operate entirely within the customer's controlled IT environment.

\subsubsection{User base and knowledge requirements}

The primary user base of Sovelia Core consists of mechanical engineers, product designers, production planners, and technical documentation specialists. These users interact with the system daily to access product information, manage engineering changes, release manufacturing documentation, and maintain synchronization between CAD designs and ERP data. User proficiency levels vary significantly, ranging from power users with deep system expertise to occasional users who require guidance for infrequent tasks.

A central challenge in this environment is the fragmentation of knowledge across multiple documentation sources. Users must navigate three distinct categories of documentation. First, \textbf{vendor documentation} consists of official help files, user guides, and release notes provided by Sovelia, documenting system features, configuration options, and best practices. These include \textit{Sovelia Help} \parencite{noauthor_sovelia_nodate}, an extensive online manual covering all aspects of the system, and periodic release notes in \textit{Sovelia.com} \parencite{noauthor_digital_nodate} that detail new features. Second, \textbf{customer-specific documentation} consists of e.g., internal standard operating procedures (SOPs), workflow guidelines, and customization documentation created by each customer organization to reflect their specific business processes and system configurations. This knowledge fragmentation can create inefficiencies in user onboarding, increase support burden, and hinder user productivity. New users usually struggle to locate relevant information, while support staff face repetitive inquiries that could be answered through better knowledge accessibility.

\subsubsection{Data governance and security context}

Sovelia Core implements role-based access control (RBAC) to restrict data visibility based on user roles and organizational structure. Users should only access product data relevant to their functional responsibilities, and this principle must extend to any AI assistant that queries the system's knowledge base. However, in the initial implementation phase, the RAG system is designed to offer retrieval of documents accessible to all users, with plans to implement finer-grained access controls in future iterations.

Additionally, customer organizations usually have strict policies regarding data processing and AI usage. Many customers express concerns about data privacy, algorithmic transparency, and the risk of AI hallucinations leading to incorrect guidance in engineering workflows. These concerns shape the requirements for any AI integration, including source attribution, and mechanisms to mitigate the risk of fabricated information. The aim is to make the AI assistant cite sources clearly and provide verifiable references for all generated responses.

\subsection{Case study boundaries and unit of analysis}
\label{subsec:case-boundaries}

The unit of analysis in this case study is the \emph{integrated RAG system} as implemented within a Sovelia Core customer environment. This includes the technical components (embedding models, vector database, retrieval logic, LLM interface), the documentation pipeline (ingestion, chunking, embedding generation), and the user-facing interface integrated into the PLM platform. Sovelia Core itself is an excellent environment for studying RAG integration due to its roots being in the Document Management System (DMS) domain, where effective knowledge management has been one of its core features from the start.

The case study is bounded by the following scope constraints:

\begin{itemize}
    \item \textbf{Temporal scope}: The research covers the design, implementation, and initial evaluation phases within a defined timeframe. Long-term adoption patterns and organizational change impacts are outside the scope.
    \item \textbf{Technical scope}: The study focuses on RAG architecture, on-premise deployment constraints, and documentation retrieval. General LLM capabilities, conversational AI design, and broader knowledge management systems are addressed only if they are related to the RAG implementation.
    \item \textbf{Organizational scope}: While organizational context shapes requirements and evaluation, the study does not constitute a comprehensive organizational change management analysis.
\end{itemize}

\subsection{Data collection and evaluation approach}
\label{subsec:data-collection}

Data collection in this case study combines multiple sources to formulate the findings. The primary data sources include:

\begin{itemize}
    \item \textbf{System implementation artifacts}: Architecture documentation, source code, configuration files, etc. that document the technical design decisions and implementation approach.
    \item \textbf{User feedback}: Qualitative feedback gathered through informal interviews, support channel discussions, and structured feedback sessions with pilot users.
    \item \textbf{Technical performance metrics}: Quantitative measures of retrieval accuracy, response latency, embedding quality, and system resource utilization.
\end{itemize}

However, the source code itself will not be made publicly available due to intellectual property considerations. The evaluation strategy follows the design science framework's emphasis on both \emph{technical performance} (Does the artifact function as designed?) and \emph{utility} (Does the artifact solve the identified problem?). Technical performance is assessed through established RAG evaluation metrics, while utility is evaluated mainly through user feedback.


%@TODO: Write this chapter
\chapter{System Design and Implementation}
\label{ch:system-design-and-implementation}

\textit{In this chapter, we detail the design and implementation of the AI assistant integrated with the Product Lifecycle Management (PLM) system. We discuss the architecture, technologies used, and the development process.}

\section{RAG system architecture}
\label{sec:rag-system-architecture}

\textit{The architecture of the Retrieval-Augmented Generation (RAG) system is designed to fit within Sovelia Core's on-premises, PostgreSQL-based environment while adhering to data governance constraints. The system consists of several key components:}

\dots

\section{Documentation Pipeline Design}
\label{sec:documentation-pipeline-design}

\textit{The documentation pipeline is engineered to ensure reliability and maintainability through a series of well-defined stages: parsing, chunking, embedding, and retiring. Each stage is designed to handle specific tasks efficiently while allowing for easy updates and modifications.}


\chapter{Evaluation and Results}
\label{ch:evaluation-and-results}

In this chapter, we present the evaluation of the AI assistant integrated with the Product Lifecycle Management (PLM) system. We discuss the methodologies used for evaluation, the metrics considered, and the results obtained from various tests and user feedback.

\section{Testing and Validation of Chatbot Functionality}
\label{sec:testing-and-validation-of-chatbot-functionality}

The chatbot functionality was tested through a series of scenarios that mimic real-world interactions within the PLM environment. These tests included both automated scripts and manual interactions by users familiar with the PLM system. The chatbot's ability to understand and respond to queries related to product data, documentation retrieval, and workflow assistance was evaluated.

\section{User Feedback}
\label{sec:user-feedback}

User feedback was collected through surveys and interviews with a group of PLM system users who interacted with the AI assistant. The feedback focused on the usability, effectiveness, and overall satisfaction with the chatbot. Users were asked to rate their experience on various aspects, including response accuracy, ease of use, and the helpfulness of the information provided.


%@TODO: Write this chapter
\chapter{Conclusion}
\label{ch:conclusion}

\textit{In this thesis, we have explored the integration of AI assistants with Product Lifecycle Management (PLM) systems to enhance their functionality and user experience. Our research has demonstrated that such integrations can lead to significant improvements in efficiency, decision-making, and collaboration among stakeholders.}


%%%%% Bibliography/references.

\printbibliography[heading=bibintoc]

%%%%% Appendices.

\begin{appendices}

%
% appendices/index.tex
%
% Input your appendices here.
%

\chapter{Example appendix}
\label{app:example}

This text is intended as an example for making an appendix in this document template. A little longer one makes it look like a proper paragraph.



\end{appendices}

%%%%% PhD Publications

\publicationmatter

%
% publications/index.tex
%
% Input your publications here using the two publication commands.
% Comment these out if you are not writing a PhD dissertation,
% or your PhD dissertation is a monograph instead of a compilation
% thesis.
%

\nonFreePublication{article1}{publications/article-1.pdf}

\nonFreePublication{article2}{publications/article-2.pdf}

\licensedPublication{article3}{publications/article-3.pdf}{Some~License}


\end{document}
