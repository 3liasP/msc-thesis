%%%%%%%%%%%%%%%%%%%%%%%%%%%%%%%%%%%%%%%%%%%%%%%%%%%%%%%%%%%
%% Congratulations, you've made an excellent choice
%% of writing your Tampere University thesis using
%% the LaTeX system. This document attempts to be
%% as complete a template as possible to let you focus
%% on the most important part: the writing itself.
%% Thus the details regarding the visual appearance
%% and even structure have already been worked out
%% for you!
%%
%% I sincerely hope you will find this template useful
%% in completing your thesis project. I've tried to
%% add comments (followed by the % sign) to clarify
%% the structure and purpose of some of the commands.
%% Most of the magic happens in the file tauthesis.cls,
%% which you are more than welcome to take a look at.
%% Just refrain from editing it in the most crucial
%% versions of the thesis!
%%
%% I wish you and your thesis project the best of luck!
%% If this template causes you trouble along the way
%% or if you've any suggestions for improving it,
%% please be in contact through GitHub
%% (<URL HERE>)
%%
%% Yours,
%%
%% Ville Koljonen
%%
%% PS. This template or its associated class file don't
%% come with a warranty. The content is provided as is,
%% without even the implied promise of fitness to the
%% mentioned purpose. You, as the author of the thesis,
%% are responsible for the entire work, including the
%% provided material. No one else is liable to you for
%% any damage inflicted on you or your thesis, were it
%% caused by using this template or not.
%%%%%%%%%%%%%%%%%%%%%%%%%%%%%%%%%%%%%%%%%%%%%%%%%%%%%%%%%%%

%%%%% NOTICE %%%%%
%% Please read through the entire template
%% (files under ./tex) to find all instructions.
%% It is possible that the attached pdf files
%% do not include the latest information.
%%%%%%%%%%%%%%%%%%

%%%%% INSTRUCTIONS FOR COMPILING THE DOCUMENT %%%%%
%% Overleaf: just click Recompile.
%% Terminal:
%%  1. pdflatex main.tex
%%  2. makeindex -s main.ist -t main.glg -o main.gls main.glo
%%  3. biber main
%%  4. pdflatex main.tex
%%  5. pdflatex main.tex
%% Similar sequence of commands is also required
%% in LaTeX specific editors.
%%%%%%%%%%%%%%%%%%%%%%%%%%%%%%%%%%%%%%%%%%%%%%%%%%%

%%% Set PDF version before doing anything else.

%%% set-pdf-version.tex
%
% This file is loaded by main.tex before any other operations, so that PDF
% version is set correctly for accessibility features.
%

\def\mypdfminorversion{7}

\directlua {
    if pdf.getminorversion() \string~= \mypdfminorversion then
        if (status.pdf_gone and status.pdf_gone > 0)
        or (status.pdf_ptr and status.pdf_ptr > 0)
        then
            tex.error("PDF version cannot be changed anymore.")
        else
            pdf.setminorversion(\mypdfminorversion)
        end
    end
}


%%%%% METADATA %%%%%
%
% Always keep the following metadata up to date! This is important for your
% PDF file to comply to accessibility standards. (And yes, this information
% must remain here, before \documentclass[...]{...}.)

%
% metadata.tex
%
% Fill in your document metadata in this file.
% This is included into the file main.tex for you.
%
% Define your citation options here.
% Valid values for style and sorting options
% can be found in the BibLaTeX manual: https://ctan.org/pkg/biblatex.
%
% Valid values for thesis types are bachelor, master, licentiate and doctor for enthesistype,
% and kandidaatti, maisteri, diplomi, lisensiaatti and tohtori for fithesistype.
% The Finnish end English thesis types need to match.
%
% The rest of the inputs should be rather self-explanatory.

% This declares to LaTeX that it should use modern internale LaTeX
% features not available in LaTeX2e.

\DocumentMetadata{
  pdfversion=1.7,
  pdfstandard=A-2b,
  lang=fi,
  uncompress,
}

\documentclass[
  author={Firstname Lastname},
  citationsorting=nyt,
  citationstyle=ieee,
  enfacultyname={Faculty},
  enkeywords={keyword1, keyword2, ...},
  enprogrammename={Thesis programme},
  ensubtitle={Descriptive subtitle},
  enthesistype=doctor,
  entitle={Thesis template},
  fifacultyname={Tiedekunta},
  fikeywords={avainsana1, avainsana2, ...},
  fiprogrammename={Tutkinto-ohjelman nimi},
  fisubtitle={Kuvaava alaotsikko},
  fithesistype={tohtori},
  fititle={Opinnäytetyöpohja},
  includePublications=true,
  mainlanguage={finnish},
  subject={Metatietoihin tuleva työn lyhyt kuvaus},
]{tauthesis}


%%% preamble.tex
%
% This file is for including LaTeX libraries or packages and defining your own
% commands.
%
% NOTE: The glossaries package loaded by tauthesis.cls throws a warning: No
% language module detected for 'finnish'. You can safely ignore this. All
% other warnings should be taken care of, before your thesis is submitted!

%%%%% Your packages.
%
% Before adding packages, see if they can be found in tauthesis.cls already.
% If you're not sure that you need a certain package, don't include it in the
% document! This can dramatically reduce compilation time.

% Graphs
% \usepackage{pgfplots}
% \pgfplotsset{compat=1.15}

% Subfigures and wrapping text
% \usepackage{subcaption}

%% Theorem environments and their numbering.
%
% Define both English and Finnish theorem types. These all follow the same
% counter. See the documentation of amsthm to see how these can be changed to
% suit your needs, if necessary.
%

\usepackage{amsthm}

\theoremstyle{definition}

\newtheorem{definition}{Definition}[chapter]
\newtheorem{theorem}[definition]{Theorem}
\newtheorem{lemma}[definition]{Lemma}
\newtheorem{corollary}[definition]{Corollary}
\newtheorem{example}[definition]{Example}

\newtheorem{maaritelma}[definition]{Määritelmä}
\newtheorem{lause}[definition]{Lause}
\newtheorem{apulause}[definition]{Apulause}
\newtheorem{seurauslause}[definition]{Seurauslause}
\newtheorem{esimerkki}[definition]{Esimerkki}

% Mathematics packages
\usepackage{mathtools, amssymb}
%\usepackage{bm}

% Chemistry packages
% \usepackage{chemfig}
% \usepackage[version=4]{mhchem}

% Text hyperlinking
% \usepackage{hyperref}
% \hypersetup{hidelinks}

% (SI) unit handling
\usepackage{siunitx}

\sisetup{
    detect-all,
    math-sf=\mathrm,
    exponent-product=\cdot,
    output-decimal-marker={,} % for theses in FINNISH!
}

%% For code listings.

\usepackage{listings}

% This global code listing configuration is required for automatically
% replacing special characters with corresponding LaTeX commands, in code files
% included with \lstinputlisting. Without it, letters like 'ä' in code file
% comments will result in a LaTeX errors.
%
% Source: https://tex.stackexchange.com/a/574950.

\lstset{
    inputencoding = utf8,  % Input encoding
    extendedchars = true,  % Extended ASCII
    literate      =        % Support additional characters
        {á}{{\'a}}1  {é}{{\'e}}1  {í}{{\'i}}1 {ó}{{\'o}}1  {ú}{{\'u}}1
        {Á}{{\'A}}1  {É}{{\'E}}1  {Í}{{\'I}}1 {Ó}{{\'O}}1  {Ú}{{\'U}}1
        {à}{{\`a}}1  {è}{{\`e}}1  {ì}{{\`i}}1 {ò}{{\`o}}1  {ù}{{\`u}}1
        {À}{{\`A}}1  {È}{{\`E}}1  {Ì}{{\`I}}1 {Ò}{{\`O}}1  {Ù}{{\`U}}1
        {ä}{{\"a}}1  {ë}{{\"e}}1  {ï}{{\"i}}1 {ö}{{\"o}}1  {ü}{{\"u}}1
        {Ä}{{\"A}}1  {Ë}{{\"E}}1  {Ï}{{\"I}}1 {Ö}{{\"O}}1  {Ü}{{\"U}}1
        {â}{{\^a}}1  {ê}{{\^e}}1  {î}{{\^i}}1 {ô}{{\^o}}1  {û}{{\^u}}1
        {Â}{{\^A}}1  {Ê}{{\^E}}1  {Î}{{\^I}}1 {Ô}{{\^O}}1  {Û}{{\^U}}1
        {œ}{{\oe}}1  {Œ}{{\OE}}1  {æ}{{\ae}}1 {Æ}{{\AE}}1  {ß}{{\ss}}1
        {ẞ}{{\SS}}1  {ç}{{\c{c}}}1 {Ç}{{\c{C}}}1 {ø}{{\o}}1  {Ø}{{\O}}1
        {å}{{\aa}}1  {Å}{{\AA}}1  {ã}{{\~a}}1  {õ}{{\~o}}1 {Ã}{{\~A}}1
        {Õ}{{\~O}}1  {ñ}{{\~n}}1  {Ñ}{{\~N}}1  {¿}{{?`}}1  {¡}{{!`}}1
        {°}{{\textdegree}}1 {º}{{\textordmasculine}}1 {ª}{{\textordfeminine}}1
        {£}{{\pounds}}1  {©}{{\copyright}}1  {®}{{\textregistered}}1
        {«}{{\guillemotleft}}1  {»}{{\guillemotright}}1  {Ð}{{\DH}}1  {ð}{{\dh}}1
        {Ý}{{\'Y}}1    {ý}{{\'y}}1    {Þ}{{\TH}}1    {þ}{{\th}}1    {Ă}{{\u{A}}}1
        {ă}{{\u{a}}}1  {Ą}{{\k{A}}}1  {ą}{{\k{a}}}1  {Ć}{{\'C}}1    {ć}{{\'c}}1
        {Č}{{\v{C}}}1  {č}{{\v{c}}}1  {Ď}{{\v{D}}}1  {ď}{{\v{d}}}1  {Đ}{{\DJ}}1
        {đ}{{\dj}}1    {Ė}{{\.{E}}}1  {ė}{{\.{e}}}1  {Ę}{{\k{E}}}1  {ę}{{\k{e}}}1
        {Ě}{{\v{E}}}1  {ě}{{\v{e}}}1  {Ğ}{{\u{G}}}1  {ğ}{{\u{g}}}1  {Ĩ}{{\~I}}1
        {ĩ}{{\~\i}}1   {Į}{{\k{I}}}1  {į}{{\k{i}}}1  {İ}{{\.{I}}}1  {ı}{{\i}}1
        {Ĺ}{{\'L}}1    {ĺ}{{\'l}}1    {Ľ}{{\v{L}}}1  {ľ}{{\v{l}}}1  {Ł}{{\L{}}}1
        {ł}{{\l{}}}1   {Ń}{{\'N}}1    {ń}{{\'n}}1    {Ň}{{\v{N}}}1  {ň}{{\v{n}}}1
        {Ő}{{\H{O}}}1  {ő}{{\H{o}}}1  {Ŕ}{{\'{R}}}1  {ŕ}{{\'{r}}}1  {Ř}{{\v{R}}}1
        {ř}{{\v{r}}}1  {Ś}{{\'S}}1    {ś}{{\'s}}1    {Ş}{{\c{S}}}1  {ş}{{\c{s}}}1
        {Š}{{\v{S}}}1  {š}{{\v{s}}}1  {Ť}{{\v{T}}}1  {ť}{{\v{t}}}1  {Ũ}{{\~U}}1
        {ũ}{{\~u}}1    {Ū}{{\={U}}}1  {ū}{{\={u}}}1  {Ů}{{\r{U}}}1  {ů}{{\r{u}}}1
        {Ű}{{\H{U}}}1  {ű}{{\H{u}}}1  {Ų}{{\k{U}}}1  {ų}{{\k{u}}}1  {Ź}{{\'Z}}1
        {ź}{{\'z}}1    {Ż}{{\.Z}}1    {ż}{{\.z}}1    {Ž}{{\v{Z}}}1
}

%%%%% Your commands.

% Print verbatim LaTeX commands
\newcommand{\verbcommand}[1]{\texttt{\textbackslash #1}}

% Command for formatting code.

\newcommand\code[1]{\texttt{#1}}

% A delimiter command for the norm of a vector with mathtools.

\DeclarePairedDelimiter\norm{\lVert}{\rVert}

% Basic theorems in Finnish and in English.
% Remove [chapter] if you wish a simply
% running enumeration.
% \newtheorem{lause}{Lause}[chapter]
% \newtheorem{theorem}[lause]{Theorem}

% \newtheorem{apulause}[lause]{Apulause}
% \newtheorem{lemma}[lause]{Lemma}

% Use these versions for individually
% enumerated lemmas
% \newtheorem{apulause}{Apulause}[chapter]
% \newtheorem{lemma}{Lemma}[chapter]

% Definition style
% \theoremstyle{definition}
% \newtheorem{maaritelma}{Määritelmä}[chapter]
% \newtheorem{definition}[maaritelma]{Definition}
% examples in this style

%%%%% Glossary information.

% Use the following lines ONLY if you need more
% than one glossary. The first argument specifies
% a type label for the glossary and the second
% the displayed name.
% \newglossary*{symbs}{Symbols}
% \newglossary{label}{Displayed name}
% ...

\makeglossaries

% Use this line if using the default glossary.
% Otherwise comment out.

\loadglsentries[main]{tex/sanasto.tex}

% Use this line if using more than one glossary.
% Otherwise comment out.
% \loadglsentries[symbs]{tex/sanasto2.tex}

%%%%% Citation information.

% Commonly used bibliography modifications.
% Feel free to play around with them.

%\ExecuteBibliographyOptions{%
%sorting=none,
%maxbibnames=99,
%maxcitenames=2,
%giveninits=true,
%uniquename=init,
%sortcites,
%sortlocale=fin}

%\DeclareNameAlias{sortname}{last-first}
%\DeclareNameAlias{author}{last-first}

%\DeclareFieldFormat[%
%    article,inbook,incollection,inproceedings,
%    patent,thesis,unpublished]{citetitle}{#1\isdot}
%\DeclareFieldFormat[%
%    article,inbook,incollection,inproceedings,
%    patent,thesis,unpublished]{title}{#1\isdot}
%\DeclareFieldFormat{pagetotal}{#1 \bibstring{page}}

%\AtBeginBibliography{\renewcommand*{\makelabel}[1]{#1\hss}}

%\DefineBibliographyExtras{english}{\let\finalandcomma=\empty}

\addbibresource{tex/references.bib}
 % You can add packages and define new commands in this file.

\begin{document}

%%%%% FRONT MATTER %%%%%

\frontmatter

%%%%% Thesis information and title page.

% Enable the use of @ character in command names.

\makeatletter

% The titles of the work. If there is no subtitle, leave the \myfisubtitle or
% \myensubtitle command arguments empty. Pass the title in the primary
% language as the first argument and its translation to the secondary language
% as the second.

\if@langenglish

    \title{\myentitle}{\myfititle}

\else

    \title{\myfititle}{\myentitle}

\fi

\if@langenglish

    \subtitle{\myensubtitle}{\myfisubtitle}

\else

    \subtitle{\myfisubtitle}{\myensubtitle}

\fi

% The author name.

\author{\myauthor}

% The examiner information. If your work has multiple examiners, replace with
%
%   \examiner[<label>]{<name> \\ <name>}
%
% where <label> is an appropriate (plural) label, e.g. Examiners or
% Tarkastajat, and <name>s are replaced by the examiner names, each on their
% separate line.

%\examiner{\myexaminers}

% The finishing date of the thesis (YYYY-MM-DD).

\finishdate{\myyear}{\mymonth}{\myday}

% The type of the thesis (e.g. Kandidaatintyö or Master of Science Thesis) in
% the primary and the secondary languages of the thesis.

\if@langenglish

    \thesistype{\myenthesistype}{\myfithesistype}

\else

    \thesistype{\myfithesistype}{\myenthesistype}

\fi

% The faculty and degree programme names in the primary and the secondary
% languages of the thesis, respectively.

\if@langenglish

    \facultyname{\myenfacultyname}{\myfifacultyname}

\else

    \facultyname{\myfifacultyname}{\myenfacultyname}

\fi

\if@langenglish

    \programmename{\myenprogrammename}{\myfiprogrammename}

\else

    \programmename{\myfiprogrammename}{\myenprogrammename}

\fi

% The keywords of the thesis in the primary and the secondary languages of the
% thesis.

\if@langenglish

    \keywords{\myenkeywords}{\myfikeywords}

\else

    \keywords{\myfikeywords}{\myenkeywords}
\fi

% Make @ a regular letter again.

\makeatother

% Actually generate the title page based on the above commands.

\maketitle


%%%%% Abstracts and preface.
%
% Write the abstract(s) and the preface into a separate file for the sake of
% clarity. Pass the appropriate file name as the first argument to these
% commands. Put the \abstract in the primary language first and the
% \otherabstract in the secondary language second. Those who do not speak
% Finnish only need the first abstract. The second argument of the \preface
% command takes the place where the thesis was signed in.
%
% Edit the files tex/{use-of-ai,tekoalyn-kaytto}.tex to match your use of AI in
% generating this thesis.
%

\abstract

\otherabstract

\aidisclaimerinclusioncmd

\preface{tex/alkusanat.tex}{Tampereella}

%%%%% Table of contents.


\hypersetup{
	linkcolor=black,
}

\tableofcontents

%%%%% Lists of figures, tables, listings and terms.
%
% Print the lists of figures and/or tables. Uncomment either of these commands
% as required. Both are optional, but if there are many important
% figures/tables, listing them may be a good idea.

% \listoffigures
% \listoftables
% \lstlistoflistings

% Misc stuff related to how the glossary is displayed. You can especially
% tweak the lengths to suit you!

\glsaddall
\setglossarystyle{taulong}
\setlength{\glsnamewidth}{0.25\textwidth}
\setlength{\glsdescwidth}{0.75\textwidth}
\renewcommand*{\glsgroupskip}{}

% Print the default glossary of abbreviations, if necessary. Otherwise comment
% out. The appropriate Finnish variant is 'Lyhenteet'

\printglossary[title={Lyhenteet ja merkinnät}]

% Print more than one glossary with these lines. Otherwise comment out.

% \printglossary[type=symbs]
% \printglossary[type=label]
% ...

%%%%% MAIN MATTER %%%%%

\mainmatter

\hypersetup{
	linkcolor=taupurple,
}

% Actually include your chapters in the file specified by this input statement.



\serialNumberPage

%%%%% Abstracts and preface.
%
% Write the abstract(s) and the preface into a separate file for the sake of
% clarity. Pass the appropriate file name as the first argument to these
% commands. Put the \abstract in the primary language first and the
% \otherabstract in the secondary language second. Those who do not speak
% Finnish only need the first abstract. The second argument of the \preface
% command takes the place where the thesis was signed in.
%
% Edit the files tex/{use-of-ai,tekoalyn-kaytto}.tex to match your use of AI in
% generating this thesis.
%

\abstract

\otherabstract

\preface{frontmatter/alkusanat.tex}{Tampereella}

%%%%% Table of contents.


\hypersetup{
	linkcolor=black,
}

\tableofcontents

%%%%% Lists of figures, tables, listings and terms.
%
% Print the lists of figures and/or tables. Uncomment either of these commands
% as required. Both are optional, but if there are many important
% figures/tables, listing them may be a good idea.

\listOfFigures

\listOfTables

\listOfListings

% Misc stuff related to how the glossary is displayed.

\glsaddall
\setglossarystyle{taulong}
\setlength{\glsnamewidth}{0.25\textwidth}
\setlength{\glsdescwidth}{0.75\textwidth}
\renewcommand*{\glsgroupskip}{}

% Print the default glossary of abbreviations, if necessary. Otherwise comment
% out. The appropriate Finnish variant is 'Lyhenteet'

\printglossary[title={Lyhenteet ja merkinnät}]

% Print more than one glossary with these lines. Otherwise comment out.

% \printglossary[type=symbs]
% \printglossary[type=label]
% ...

\listofpublications

% AI disclaimer page.

\aidisclaimerinclusioncmd


\end{document}
