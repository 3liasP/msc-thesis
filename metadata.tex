%!TEX root = main.tex
%
% metadata.tex
%
% Fill in your document metadata in this file.
% This is included into the file main.tex for you.
%
% Define your citation options here.
% Valid values for style and sorting options
% can be found in the BibLaTeX manual: https://ctan.org/pkg/biblatex.
%
% Valid values for thesis types are bachelor, master, licentiate and doctor for enthesistype,
% and kandidaatti, maisteri, diplomi, lisensiaatti and tohtori for fithesistype.
% The Finnish end English thesis types need to match.
%
% The rest of the inputs should be rather self-explanatory.

% This declares to LaTeX that it should use modern internal LaTeX
% features not available in LaTeX2e. This is also responsible for
% PDF/A generation.

\DocumentMetadata{
  pdfversion=1.7,
  pdfstandard=A-3b,
  % Make sure this matches the main document language below.
  % One of fi-FI or en-{US,UK,EN}
  lang=en-{US,UK,EN},
  % This is related to well-tagged PDFs. If you change this to phase-III or higher (please don't),
  % you will be required to manually tag most of your document according to the tagpdf package documentation.
  % Note that it has not been confirmed whether Trepo will actually accept well-tagged PDF files as of spring 2025,
  % so you might not want to do this yet until that is confirmed. Also, even phase-II seems to break PDF/A compatibility.
  %testphase={phase-III,math,table,firstaid},
  %uncompress,
}

\documentclass[
  author={Elias Peltonen},
  citationsorting={nyt},
  citationstyle={ieee},
  enfacultyname={Faculty of Information Technology and Communication Sciences},
  enkeywords={AI, Chatbot, PLM, Sovelia Core, Resource-Augmented Generation, Master's Thesis},
  enprogrammename={Master's Programme in Information Technology, Software Engineering},
  ensubtitle={A case study on Sovelia Core},
  enthesistype={master},
  entitle={Integrating resource-augmented AI chatbots into custom-deployed PLM systems},
  fifacultyname={Informaatioteknologian ja viestinnän tiedekunta},
  fikeywords={tekoäly, keskustelurobotti, PLM, Sovelia Core, resurssivahvistettu generointi, diplomityö},
  fiprogrammename={Tietotekniikan DI-ohjelma, Ohjelmistot},
  fisubtitle={Tapaustutkimus Sovelia Core-järjestelmästä},
  fithesistype={diplomi},
  fititle={Resurssivahvistettujen tekoälykeskustelurobottien integrointi räätälöityihin PLM-järjestelmiin},
  includeListOfFigures=true,
  includeListOfTables=true,
  includeListOfListings=true,
  includePublications=false,
% Make sure this matches the language given in \DocumentMetadata
  mainlanguage={english},
  subject={This thesis explores integrating a resource-augmented AI assistant (RAG chatbot) into Sovelia Core, a customizable PLM platform. The goal is to improve access to documentation and support by enabling the chatbot to answer questions using both vendor and customer materials, helping users find information and onboard more efficiently.},
]{tauthesis}
