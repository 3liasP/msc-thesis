\chapter{Research Methodology}
\label{ch:research-methodology}

%@TODO: add introduction to this chapter

\section{Design Science Methodology}
\label{sec:design-science-methodology}

Design Science Research (DSR) is a problem-solving paradigm that seeks to create and evaluate artifacts intended to solve identified organizational problems \parencite{johannesson_introduction_2021}. Unlike explanatory science, which aims to understand and explain phenomena, design science is prescriptive—it produces knowledge about how to design and build artifacts that serve human purposes. In the context of information systems research, these artifacts can take various forms, including constructs, models, methods, and instantiations.

According to \textcite{johannesson_introduction_2021}, design science research follows a structured process consisting of five key activities: (1) \emph{explicate problem}, where the practical problem and its context are identified and defined; (2) \emph{define requirements}, which establishes the criteria that the artifact must satisfy; (3) \emph{design and develop artifact}, involving the actual construction of the solution; (4) \emph{demonstrate artifact}, where the artifact's utility is shown in solving the problem; and (5) \emph{evaluate artifact}, assessing how well it meets the requirements and solves the problem.

In this thesis, we employ design science methodology to develop an AI assistant integrated with Sovelia Core, an on-premise Product Lifecycle Management (PLM) system. The practical problem we address is the knowledge accessibility challenge faced by users navigating fragmented documentation across vendor manuals, release notes, and customer-specific standard operating procedures. Our artifact—a Retrieval-Augmented Generation (RAG) system—is designed, implemented, and evaluated through iterative cycles within a live organizational context. This approach enables us to contribute both a practical solution for Sovelia Core and generalizable knowledge about RAG system integration in on-premise enterprise PLM environments.

\section{Case Study Approach}
\label{sec:case-study-approach}

%TODO: write content for this section

\textit{A case study approach is utilized to provide an in-depth understanding of the integration of the AI assistant with the PLM system. This approach allows for the exploration of real-world contexts and the examination of the interactions between the AI assistant and users within the PLM environment. Data is collected through observations, interviews, and system usage logs to assess the effectiveness and usability of the integrated solution.}
