This document template conforms to the Guide to Writing a Thesis in Tampere University \parencite{thesisguide2018}. A thesis typically includes the following parts:

\begin{enumerate}
    \item[] Title page
    \item[] Abstract in English (and in Finnish)
    \item[] Preface
    \item[] Contents
    \item[] (Lists of figures and tables)
    \item[] List of abbreviations and symbols
    \item Introduction
    \item Theoretical background
    \item Research methodology and material
    \item Results and analysis (possibly in separate chapters)
    \item Conclusions
    \item[] References
    \item[] (Appendices)
\end{enumerate}

Each of these is written as a new \verbcommand{chapter} or using an appropriate command (e.g. \verbcommand{abstract}). Read this document template and its comments carefully. The titles of chapters from 1 to 5 are provided as examples only. You should use more descriptive ones. The title page is created by inserting the relevant information into the commands near the start of the template. The table of contents lists all the numbered headings after it, but not always the preceding headings.

Introduction outlines the purpose and objectives of the presented research. The background information, utilized methods and source material are presented next at a level that is necessary to understand the rest of the text. Then comes the discussion regarding the achieved results, their significance, error sources, deviations from the expected results, and the reliability of your research. The conclusions form the most important chapter. It does not repeat the details already presented, but summarizes them and analyzes their consequences. List of references enables your reader to find the cited sources.

This document is structured as follows. Chapter \ref{ch:esitystyyli} discusses briefly the basics of writing and presentation style regarding the text, figures, tables and mathematical notations. Chapters \ref{ch:viittaustekniikat} and \ref{ch:yhteenveto} summarize the referencing basics and the whole document. Each part also features tips to solving some of the detailed issues that relate to writing in \LaTeX.