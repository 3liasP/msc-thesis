\chapter{Introduction}
\label{ch:introduction}

This chapter introduces the research topic, providing background information and outlining the motivation behind the study. It also presents the research objectives and questions that guide the investigation throughout the thesis.

\section{Background and motivation}
\label{sec:background-and-motivation}

In recent years, the integration of Artificial Intelligence (AI) assistants into various software systems has gained significant attention. AI assistants have the potential to enhance user experience, streamline workflows, and improve decision-making processes. Product Lifecycle Management (PLM) systems, which manage the entire lifecycle of a product from inception to disposal \parencite{stark_product_2015}, can particularly benefit from the integration of AI assistants. By leveraging AI technologies, PLM systems can provide users with intelligent support, automate routine tasks, and allow for better collaboration among various stakeholders. For instance, AI applications in PLM span product design, manufacturing, and service stages, offering opportunities for innovation while presenting challenges such as data governance and algorithm reliability \parencite{wang_artificial_2021}.

In the case of PLM systems, one of the key challenges is the fact that they often handle vast amounts of documentation, including standard operating procedures (SOPs), manuals, and technical specifications. Managing this documentation effectively is important for ensuring that users can access the information they need when they need it. However, the explosive growth of unstructured and heterogeneous data in PLM systems can lead to fragmentation and integration difficulties \parencite{wang_artificial_2021}. This makes it challenging to maintain up-to-date and accurate documentation, which is essential for effective decision-making and operational efficiency. Because of the adaptability of PLM systems to various industries and organizational needs, high-quality documentation has a key role in user training, compliance, and knowledge transfer since there is no one-size-fits-all solution for PLM implementations \parencite{stark_product_2015-1}. While PLM systems can be deployed in various environments, they often operate in on-premises environments with strict data governance policies, which can further complicate the integration of AI assistants.

To manage and utilize this documentation effectively, there is a growing interest in developing AI assistants that can interact with PLM systems. These AI assistants can help users navigate the vast amounts of documentation, provide context-aware help for specific tasks, and allow for knowledge sharing among team members. To implement an effective context-aware AI assistant within a PLM system, it is needed to address several challenges, including data integration, natural language processing, and user interface design. One promising approach to building such AI assistants is the use of Retrieval-Augmented Generation (RAG) systems, which combine large language models (LLMs) with external knowledge bases to provide accurate and contextually relevant responses \parencite{lewis_retrieval-augmented_2021}.

%@TODO: expand this background section later with content below

\textit{I will add more information about Sovelia Core and its documentation management challenges here...}

\textit{Additionally, I will elaborate a bit on the research methodology and system design generally here...}

\section{Research objectives and questions}
\label{sec:research-objectives-and-questions}

The primary objective of this research is to develop an AI assistant that is seamlessly integrated with a PLM system, specifically Sovelia Core, to enhance user experience and improve documentation management. The AI assistant will leverage RAG techniques to provide context-aware assistance based on the extensive documentation from various sources stored within the database of the PLM system. To achieve this objective, the research will focus on the following research questions:

\begin{itemize}
	\item \textbf{RQ1 (Design)}: What RAG system architecture best fits Sovelia Core's on-prem, PostgreSQL-based environment and data governance constraints?
	\item \textbf{RQ2 (Process)}: How should customer documentation pipelines (parse $ \rightarrow $ chunk $ \rightarrow $ embed $ \rightarrow $ retire) be engineered for reliability and maintainability?
	\item \textbf{RQ3 (Quality \& Risk)}: Which risks (hallucinations, stale context, access control) are most critical, and what mitigations work in practice?
\end{itemize}

By addressing these research questions, this study aims to contribute to the development of effective AI assistants for PLM systems, ultimately enhancing user experience and improving documentation management in complex industrial environments.

In chapter~\ref{ch:literature-review}, a comprehensive overview of the relevant literature and technologies related to AI assistants, PLM systems, and RAG techniques is provided. Chapter~\ref{ch:research-methodology} outlines the research methodology employed in this study, including the design and implementation of the AI assistant. In chapter~\ref{ch:system-design-and-implementation}, the architectural choices and technical implementations made during the development of the AI assistant are detailed. In chapter~\ref{ch:evaluation-and-results}, the results of the experiments and evaluations are presented. Finally, chapter~\ref{ch:conclusion} concludes the thesis with a summary of contributions, limitations, and suggestions for future research.

%@TODO: write a proper summary of the chapters above once they are finalized

\textit{The summary above will be updated later to reflect the actual chapter titles and content once they are finalized...}
