\chapter{Introduction}
\label{ch:introduction}

This chapter introduces the research topic, providing background information and outlining the motivation behind the study. It also presents the research objectives and questions that guide the investigation.

\section{Background and Motivation}
\label{sec:background-and-motivation}

In recent years, the integration of Artificial Intelligence (AI) assistants into various software systems has gained significant attention. AI assistants have the potential to enhance user experience, streamline workflows, and improve decision-making processes. Product Lifecycle Management (PLM) systems, which manage the entire lifecycle of a product from inception to disposal, can particularly benefit from the integration of AI assistants. By leveraging AI technologies, PLM systems can provide users with intelligent support, automate routine tasks, and facilitate better collaboration among stakeholders.

\section{Research Objectives and Questions}
\label{sec:research-objectives-and-questions}

The primary objective of this research is to develop an AI assistant that is seamlessly integrated with a PLM system, enhancing the overall user experience and improving decision-making processes. To achieve this objective, we have formulated the following research questions:

\begin{itemize}
	\item \textbf{RQ1 (Design)}: What RAG system architecture best fits Sovelia Core's on-prem, PostgreSQL-based environment and data governance constraints?
	\item \textbf{RQ2 (Process)}: How should customer documentation pipelines (parse $ \rightarrow $ chunk $ \rightarrow $ embed $ \rightarrow $ retire) be engineered for reliability and maintainability?
	\item \textbf{RQ3 (Quality \& Risk)}: Which risks (hallucinations, stale context, access control) are most critical, and what mitigations work in practice?
\end{itemize}
